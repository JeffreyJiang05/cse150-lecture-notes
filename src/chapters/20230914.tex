\lecturechapter{Written by Jeffrey J.}

\section{Introduction to Boolean Algebra, Continued.}
\index{Boolean Algebra}

\subsection{Basic Definitions and Examples}

\begin{definition}
    A \emph{proposition} is a statement that is either \emph{true} or \emph{false}.
\end{definition}

Examples:
\begin{itemize}
    \item ``All humans are mortal.'' Proposition that is true\footnote{Does Henrietta Lacks count as immortal? Does her cancerous cells even count as being her?}
    \item ``\(a^2+b^2+c^2=d^2\) has no solutions where \(a,b,c,d\in\mathbb{Z}\).'' Proposition that is true
    \item ``The moon is purple.'' Proposition that is false
    \item Goldbach's Conjecture: ``Every even number greater than \(2\) can be written as the sum of two primes.'' 
    Proposition that has not been proven\footnote{Fun fact. In 1937, Ivan Vinogradov proved that every large enough
    odd number can be written as the sum of three odd primes. In 2013, Harald Helfgott proved that every odd number greater than \(5\)
    can be written as the sum of three primes. This is called Goldbach's Weak Conjecture.}
    \item ``Did you do the dishes?'' Not a proposition. It is a question.
    \item ``Play Mario with me.'' Not a proposition. It is a command.
\end{itemize}

There are three fundamental operations in boolean algebra:\index{Boolean Algebra!Operations}
\begin{itemize}
    \item \emph{Conjunction}. This is ``logical and'' and it is denoted with \(\land\).
    \item \emph{Disjunction}. This is ``logical or'' and it is denoted with \(\lor\).
    \item \emph{Negation}. This is ``logical not'' and it is denoted with \(\lnot, \bar{x}, !x\) where \(x\) is a proposition.
\end{itemize}

The truth tables of these basic operation are shown below:

\begin{figure}[H]
    \centering
    Truth Table for Conjunction

    \begin{tabular}{ccc}
        \(x\) & \(y\) & \(x\land y\) \\
        \hline
        \(T\) & \(T\) & \(T\) \\
        \(T\) & \(F\) & \(F\) \\
        \(F\) & \(T\) & \(F\) \\
        \(F\) & \(F\) & \(F\)
    \end{tabular}
\end{figure}

\begin{figure}[H]
    \centering
    Truth Table for Disjunction

    \begin{tabular}{ccc}
        \(x\) & \(y\) & \(x\lor y\) \\
        \hline
        \(T\) & \(T\) & \(T\) \\
        \(T\) & \(F\) & \(T\) \\
        \(F\) & \(T\) & \(T\) \\
        \(F\) & \(F\) & \(F\)
    \end{tabular}
\end{figure}

\begin{figure}[H]
    \centering
    Truth Table for Negation

    \begin{tabular}{cc}
        \(x\) & \(\lnot x\) \\
        \hline
        \(T\) & \(F\) \\
        \(F\) & \(T\) 
    \end{tabular}
\end{figure}

These fundamental operators give rise to more complex operations:

\begin{definition}
    \emph{Material conditional}, or \emph{implication}, represents the idea of ``implies.'' Given two propositions \(a,b\), then \(a\rightarrow b\) means that \(a\) implies \(b\)
    or ``if \(a\), then \(b\).'' The statement is true unless \(a\) is true and \(b\) is false since \(a\) being true implies that \(b\) is also true. \(a\) is called the
    \emph{antecedent} and \(b\) is called the \emph{consequent}.   
\end{definition}

\begin{figure}[H]
    \centering
    True Table for Material Conditional
    
    \begin{tabular}{ccc}
        \(a\) & \(b\) & \(a\rightarrow b\) \\
        \hline
        \(T\) & \(T\) & \(T\) \\
        \(T\) & \(F\) & \(F\) \\
        \(F\) & \(T\) & \(T\) \\
        \(F\) & \(F\) & \(T\)
    \end{tabular}
\end{figure}

\textbf{A note:} It may seem paradoxical how \(a\rightarrow b\) is true when \(a\) is false, but the true value is stating that the statement ``\(a\) implies \(b\)''
is true, not that the operands are true. The statement states that \(b\) is true when \(a\) is true. It does not state anything about what \(b\) is when \(a\) is false.
Therefore, the whole statement is true.

\bigskip
\begin{definition}
    The \emph{converse} of conditional \(a\rightarrow b\) is \(b\rightarrow a\) given propositions \(a,b\).
\end{definition}

\begin{proposition}
    \(a\rightarrow b\) is not equivalent to \(b\rightarrow a\), i.e.\ they do not always evaluate to the same truth value.
\end{proposition}
\begin{proof}
    The truth table of the two reveals how the converse of a implication is not equivalent to the implication:
    \begin{figure}[H]
        \centering
        \begin{tabular}{cccc}
            \(a\) & \(b\) & \(a\rightarrow b\) & \(b\rightarrow a\) \\
            \hline
            \(T\) & \(T\) & \(T\) & \(T\) \\
            \(T\) & \(F\) & \(F\) & \(T\) \\
            \(F\) & \(T\) & \(T\) & \(F\) \\
            \(F\) & \(F\) & \(T\) & \(T\)
        \end{tabular}
    \end{figure}
\end{proof}

Example:
\begin{itemize}
    \item ``If it is my birthday, then I am eating cake,'' is a conditional which is true. However, the converse is ``If I am eating cake, it is my birthday.''
    The converse is not necessarily true. For example, what if it is someone else's birthday, and I am eating their birthday cake. Eating the cake does not 
    imply that it is my birthday. 
\end{itemize}

\bigskip
\begin{definition}
    A \emph{vacuous conditional} is a statement that is true because the antecedent is false. 
\end{definition}

Example:
\begin{itemize}
    \item ``If the moon is purple, then the moon is made of cheese.'' This is a vacuous conditional because the ``the moon is purple'' is false. The 
    conditional is true, however.
\end{itemize}

\begin{definition}
    The \emph{contrapositive} of a material conditional \(a\rightarrow b\) is \(\lnot b\rightarrow\lnot a\).
\end{definition}

\begin{proposition}
    A material conditional \(a\rightarrow b\) is equivalent to its contrapositive \newline\(\lnot b\rightarrow\lnot a\).
\end{proposition}
\begin{proof}
    The truth table of the two reveals how the contrapositive of a implication is equivalent to the implication:
    \begin{figure}[H]
        \centering
        \begin{tabular}{cccc}
            \(a\) & \(b\) & \(a\rightarrow b\) & \(\lnot b\rightarrow\lnot a\) \\
            \hline
            \(T\) & \(T\) & \(T\) & \(T\) \\
            \(T\) & \(F\) & \(F\) & \(F\) \\
            \(F\) & \(T\) & \(T\) & \(T\) \\
            \(F\) & \(F\) & \(T\) & \(T\)
        \end{tabular}
    \end{figure}
\end{proof}

\bigskip
\begin{definition}
    \emph{Material biconditional} or \emph{equivalence} represents the idea of ``if and only if.'' Given two propositions \(a,b\), then, \(a\leftrightarrow b\) means that \(a\) implies
    \(b\) and \(b\) implies \(a\). The statement is true if \(a\) and \(b\) are both true or both false. 
\end{definition}

\begin{figure}[H]
    \centering
    True Table for Material Biconditional
    
    \begin{tabular}{ccc}
        \(a\) & \(b\) & \(a\leftrightarrow b\) \\
        \hline
        \(T\) & \(T\) & \(T\) \\
        \(T\) & \(F\) & \(F\) \\
        \(F\) & \(T\) & \(F\) \\
        \(F\) & \(F\) & \(T\)
    \end{tabular}
\end{figure}

Material biconditional is also called equivalence, denoted \(\equiv\), because it requires both the operands to be the same truth value. Additionally, if an implication
and its converse is true, then it is a biconditional. 

Perhaps the most important application of biconditionals is that all definitions are biconditionals. This is what makes a definition a definition. For example, consider
the definition of a square, i.e.\ a square is a rectangle with four equal sides. If a rectangle has four equal sides, then it is a square. Likewise, if a shape is a 
square, then it is a rectangle with four equal sides. If a definition is presented as a conditional, its converse is also true. All definitions are biconditionals.

\bigskip
\begin{definition}
    The \emph{exclusive or}, or \emph{XOR}, is true when its operands have different truth values. The symbol of XOR is typically \(\oplus\).
\end{definition}

\begin{figure}[H]
    \centering
    True Table for Exclusive Or
    
    \begin{tabular}{ccc}
        \(a\) & \(b\) & \(a\oplus b\) \\
        \hline
        \(T\) & \(T\) & \(F\) \\
        \(T\) & \(F\) & \(T\) \\
        \(F\) & \(T\) & \(T\) \\
        \(F\) & \(F\) & \(F\)
    \end{tabular}
\end{figure}

\bigskip
\begin{definition}
    A proposition is \emph{satisfiable} if some setting of truth values makes the proposition true.\footnote{Satisfiability is a NP-complete problem}
\end{definition}

\bigskip
\begin{definition}
    A proposition is a \emph{tautology} if under all settings of truth values, the proposition is true.
\end{definition}

Likewise, if all possible settings of truth values makes a proposition false, then it is called a \emph{contradiction}.

\bigskip
\begin{definition}
    A \emph{predicate} is a proposition whose truth value depends on one or more variables. In this sense, it is a function that outputs a truth value.
\end{definition}

Examples:
\begin{itemize}
    \item \(P(n)=\) ``\(n\) is even''. Then, \(P(2)=\text{true}\), \(P(1)=\text{false}\).
    \item \(P(n)=\) ``\(n\) is food''. Then, \(P(\text{Apple})=\text{true}\), \(P(\text{Table})=\text{false}\)
\end{itemize}

\subsection{Properties in Boolean Algebra}
\index{Boolean Algebra!Properties}

All of the following may be proven using truth tables or with previous conclusions. The proofs are not shown, however.
We will also denote true as \(1\) and false as \(0\) as is convention.\footnote{I'm just too lazy to type true and false every time.}

\begin{proposition} 
    \textbf{Basic Properties of Disjunction}
    \begin{itemize}
        \item Disjunction is associative, i.e.\ \((A\lor B)\lor B\equiv A\lor(B\lor C)\) for all propositions \(A,B,C\).
        \item Disjunction is commutative, i.e.\ \(A\lor B\equiv B\lor A\) for all propositions \(A,B\).
        \item The identity of disjunction is \(0\), i.e.\ \(A\lor0\equiv 0\lor A\equiv A\) for all proposition \(A\).
        \item The annihilator of disjunction is \(1\), i.e.\ \(A\lor1\equiv 1\lor A\equiv 1\) for all proposition \(A\).
    \end{itemize}
\end{proposition}

\bigskip
\begin{proposition}
    \textbf{Basic Properties of Conjunction}
    \begin{itemize}
        \item Conjunction is associative, i.e.\ \((A\land B)\land B\equiv A\land(B\land C)\) for all propositions \(A,B,C\).
        \item Conjunction is commutative, i.e.\ \(A\land B\equiv B\land A\)  for all propositions \(A,B\).
        \item The identity of conjunction is \(1\), i.e.\ \(A\land1\equiv 1\land A\equiv A\) for all proposition \(A\).
        \item The annihilator of conjunction is \(0\), i.e.\ \(A\land0\equiv 0\land A\equiv 0\) for all proposition \(A\).
    \end{itemize}
\end{proposition}

\bigskip
\begin{proposition}
    \textbf{Distributivity of Disjunction and Conjunction}. 
    \newline For all propositions \(A,B,C\):
    \begin{itemize}
        \item \(A\lor(B\land C)\equiv (A\lor B)\land(A\lor C)\)
        \item\(A\land(B\lor C)\equiv (A\land B)\lor(A\land C)\)
    \end{itemize}
\end{proposition}

\bigskip
\begin{proposition}
    \textbf{Complementation of Disjunction and Conjunction}
    \newline For all propositions \(A\):
    \begin{itemize}
        \item \(A\lor\lnot A\equiv 1\)
        \item \(A\land\lnot A\equiv 0\)
    \end{itemize}
\end{proposition}

\bigskip
\begin{proposition}
    Double negation states that for all proposition \(A\) that \(\lnot\lnot A\equiv A\).
\end{proposition}

\bigskip
\begin{proposition}
    \textbf{De Morgan's Law}
    \newline For all propositions \(A,B\):
    \begin{itemize}
        \item \(\lnot(A\lor B)\equiv\lnot A\land\lnot B\)
        \item \(\lnot(A\land B)\equiv\lnot A\lor\lnot B\)
    \end{itemize}
\end{proposition}

\bigskip
\begin{proposition}
    For all propositions \(A,B\):
    \begin{itemize}
        \item \(A\land(A\lor B)\equiv A\)
        \item \(A\lor(A\land B)\equiv A\)
    \end{itemize}
\end{proposition}

\bigskip
\begin{proposition}
    For all propositions \(A,B\):
    \begin{itemize}
        \item \((\lnot(A\equiv B))\equiv (A\oplus B)\)
        \item \((\lnot(A\oplus B))\equiv (A\equiv B)\)
    \end{itemize}    
\end{proposition}

\bigskip
An application of the above properties:
\begin{itemize}
    \item Given \texttt{if (x >= 0 \&\& (x < 0 || y))}. Let \(A=\) (\texttt{x>=0}) and \(B=\)\texttt{ y}. Note that \texttt{x<0} is equivalent to \(\lnot A\).
    Therefore, the code can be rewritten as \(A\land(\lnot A\lor B)\) in boolean algebra. Apply the distributive property for conjunction,
    \((A\land\lnot A)\lor(A\land B)\). Note the annihilator property for conjunction, \(0\lor(A\land B)\). Now apply the identity for disjunction,
    \(A\land B\). In conclusion, the expression above simplifies to \(A\land B\) or \texttt{x >= 0 \&\& y}.\footnote{In class, we proved a similar example
    using a truth table. However, knowing these properties allows you to simplify the statements without using a truth table}
\end{itemize}