\lecturechapter{Written by Jeffrey J.}

\section{Set Theory: Ordering and Relations}
\index{Set Theory!Relations}

\subsection{More on Ordering}

Given relation \(R\) on set \(S\) that is an order, the relation is strict if \(\forall\:a\in S\:(a,a)\not\in R\). This means that there are no
``self-loops.'' The relation \(R\) is strict if the relation is reflexive, i.e.\ \(\forall\:a\in S\:(a,a)\in R\). This means for all nodes, there
are ``self-loops.'' 

Relation \(R\) on set \(S\) is an order if there is a way to write the nodes so that all edges except self-loops point in the same direction. This
implies that there are some sort of ordering to the nodes since they are be arranged in such. This order is a total order\index{Set Theory!Total Order} 
if there is only one possible ordering. In contrast, the order is a partial order\index{Set Theory!Partial Order} if there are more than one possible 
orderings. 

\begin{figure}[H]
    \centering
    \textbf{Example of Strict Partial Order}

    \resizebox{12em}{12em}{
        \begin{tikzpicture}[line cap=round,line join=round,>=triangle 45,x=1.0cm,y=1.0cm]
            \clip(-2.,-7.) rectangle (12.,7.);
            \draw [->,shorten <=3.pt,shorten >=3.pt,line width=2.pt] (0.,0.) circle (1.cm);
            \draw [->,shorten <=3.pt,shorten >=3.pt,line width=2.pt] (5.,5.) circle (1.cm);
            \draw [->,shorten <=3.pt,shorten >=3.pt,line width=2.pt] (5.,-5.) circle (1.cm);
            \draw [->,shorten <=3.pt,shorten >=3.pt,line width=2.pt] (10.,0.) circle (1.cm);
            \draw [->,shorten <=3.pt,shorten >=3.pt,line width=2.pt] (5.,4.)-- (9.,0.);
            \draw [->,shorten <=3.pt,shorten >=3.pt,line width=2.pt] (5.,4.)-- (5.,-4.);
            \draw [->,shorten <=3.pt,shorten >=3.pt,line width=2.pt] (5.,4.)-- (1.,0.);
            \draw [->,shorten <=3.pt,shorten >=3.pt,line width=2.pt] (1.,0.)-- (5.,-4.);
            \draw [->,shorten <=3.pt,shorten >=3.pt,line width=2.pt] (9.,0.)-- (5.,-4.);
            \draw (4.6,5.4) node[anchor=north west] {\LARGE$A$};
            \draw (9.6,0.4) node[anchor=north west] {\LARGE$B$};
            \draw (4.6,-4.7) node[anchor=north west] {\LARGE$C$};
            \draw (-0.3,0.4) node[anchor=north west] {\LARGE$D$};
        \end{tikzpicture}
    }
    
    In this example, the two possible orderings are \(A\prec B\prec C\) or \(A\prec D\prec C\).
\end{figure}
\begin{figure}[H]
    \centering
    \textbf{Example of Strict Total Order}

    \resizebox{12em}{12em}{
        \begin{tikzpicture}[line cap=round,line join=round,>=triangle 45,x=1.0cm,y=1.0cm]
            \clip(-2.,-7.) rectangle (12.,7.);
            \draw [->,shorten <=3.pt,shorten >=3.pt,line width=2.pt] (0.,0.) circle (1.cm);
            \draw [->,shorten <=3.pt,shorten >=3.pt,line width=2.pt] (5.,5.) circle (1.cm);
            \draw [->,shorten <=3.pt,shorten >=3.pt,line width=2.pt] (5.,-5.) circle (1.cm);
            \draw [->,shorten <=3.pt,shorten >=3.pt,line width=2.pt] (10.,0.) circle (1.cm);
            \draw [->,shorten <=3.pt,shorten >=3.pt,line width=2.pt] (5.,4.)-- (9.,0.);
            \draw [->,shorten <=3.pt,shorten >=3.pt,line width=2.pt] (5.,4.)-- (5.,-4.);
            \draw [->,shorten <=3.pt,shorten >=3.pt,line width=2.pt] (5.,4.)-- (1.,0.);
            \draw [->,shorten <=3.pt,shorten >=3.pt,line width=2.pt] (1.,0.)-- (9.,0.);
            \draw [->,shorten <=3.pt,shorten >=3.pt,line width=2.pt] (1.,0.)-- (5.,-4.);
            \draw [->,shorten <=3.pt,shorten >=3.pt,line width=2.pt] (9.,0.)-- (5.,-4.);
            \draw (4.6,5.4) node[anchor=north west] {\LARGE$A$};
            \draw (9.6,0.4) node[anchor=north west] {\LARGE$B$};
            \draw (4.6,-4.7) node[anchor=north west] {\LARGE$C$};
            \draw (-0.3,0.4) node[anchor=north west] {\LARGE$D$};
        \end{tikzpicture}
    }

    In this example, there is only one possible ordering which is \(A\prec D\prec B\prec C\).
\end{figure}

\textbf{Note}: The relation may be a total or partial order depending on the the set the relation operates on. For example,
consider the relation \(R=``\le"\) on the set \(S=\mathbb{Z}\). This is a total relation because the order is fixed. Between
every two integers, they have a one-directional relation between the two. However, let \(S=\text{``Set of people"}\) and relation
\(R=``is taller than"\) on set \(S\). Since there are people who are of equal height, then there are many different possible orderings
of people. This makes the relation a partial order.

\subsection{Return to Relations}

\begin{proposition}
    On set \(S\), there are \(2^{|S|^2}\) binary relations on the set.
\end{proposition}
\begin{proof}
    The definition of the relation \(R\) on set \(S\) is that \(R\subseteq S\times S\). Therefore the total number of binary relations
    on set \(S\) is the number of the possible subsets of \(S\times S\). The set of subsets is represented by the power set of \(S\times S\) 
    or \(\mathcal{P}(S\times S)\). The total number of binary relations is \(|\mathcal{P}(S\times S)|=2^{|S\times S|}=2^{|S|^2}\). Proposition
    \(4.2\) and proposition \(4.4\) is used to derive \(2^{|S|^2}\).
\end{proof}

\bigskip
\begin{definition}
    The \emph{transitive closure}\index{Set Theory!Transitive Closure} of a binary relation \(R\) on set \(S\) is the smallest relation on \(S\) that 
    includes \(R\) and is transitive. This relation is denoted \(R^{TC}\).
\end{definition}

The transitive closure of relation \(R\) on set \(S\) but with additional elements of \(S\times S\) that makes \(R^{TC}\) transitive. Intuitively, let
\(S=\text{``locations"}\) and relation \(R=\text{``exists direct road between"}\) on set \(S\). Clearly, this relation is not necessarily transitive.
However, \(R^{TC}\) represents the relation that the two locations are reachable by road. This relation is clearly transitive and modifies \(R\).
Additionally, clearly from the definition, if relation \(R\) is transitive, \(R=R^{TC}\).

\begin{figure}[H]
    \centering
    \textbf{Example of Transitive Closure}

    Relation \(R\)

    \resizebox{20em}{5em}{
        \begin{tikzpicture}[line cap=round,line join=round,>=triangle 45,x=1.0cm,y=1.0cm]
            \clip(-3.,-2.) rectangle (13.,2.);
            \draw [line width=2.pt] (0.,0.) circle (1.cm);
            \draw [line width=2.pt] (10.,0.) circle (1.cm);
            \draw [<->,shorten <=3.pt,shorten >=3.pt,shift={(-1.,0.)},line width=2.pt]  plot[domain=1.0471975511965976:5.235987755982989,variable=\t]({1.*1.*cos(\t r)+0.*1.*sin(\t r)},{0.*1.*cos(\t r)+1.*1.*sin(\t r)});
            \draw [<->,shorten <=3.pt,shorten >=3.pt,line width=2.pt] (1.,0.)-- (9.,0.);
            \draw (9.6,0.4) node[anchor=north west] {\LARGE$B$};
            \draw (-0.3,0.4) node[anchor=north west] {\LARGE$D$};
        \end{tikzpicture}
    }

    Relation \(R^{TC}\)

    \resizebox{20em}{5em}{
        \begin{tikzpicture}[line cap=round,line join=round,>=triangle 45,x=1.0cm,y=1.0cm]
            \clip(-3.,-2.) rectangle (13.,2.);
            \draw [line width=2.pt] (0.,0.) circle (1.cm);
            \draw [line width=2.pt] (10.,0.) circle (1.cm);
            \draw [blue,<->,shorten <=3.pt,shorten >=3.pt,shift={(11.,0.)},line width=2.pt]  plot[domain=-2.094395102393194:2.094395102393194,variable=\t]({1.*1.*cos(\t r)+0.*1.*sin(\t r)},{0.*1.*cos(\t r)+1.*1.*sin(\t r)});
            \draw [<->,shorten <=3.pt,shorten >=3.pt,shift={(-1.,0.)},line width=2.pt]  plot[domain=1.0471975511965976:5.235987755982989,variable=\t]({1.*1.*cos(\t r)+0.*1.*sin(\t r)},{0.*1.*cos(\t r)+1.*1.*sin(\t r)});
            \draw [<->,shorten <=3.pt,shorten >=3.pt,line width=2.pt] (1.,0.)-- (9.,0.);
            \draw (9.6,0.4) node[anchor=north west] {\LARGE$B$};
            \draw (-0.3,0.4) node[anchor=north west] {\LARGE$D$};
        \end{tikzpicture}
    }
\end{figure}
\begin{figure}[H]
    \centering
    \textbf{Example of Transitive Closure}

    Relation \(R\)

    \resizebox{21em}{5em}{
        \begin{tikzpicture}[line cap=round,line join=round,>=triangle 45,x=1.0cm,y=1.0cm]
            \clip(-4.,-1.) rectangle (4.5,1.);
            \draw [line width=1.pt] (-3.,0.) circle (0.5cm);
            \draw [line width=1.pt] (-1.,0.) circle (0.5cm);
            \draw [line width=1.pt] (1.,0.) circle (0.5cm);
            \draw [line width=1.pt] (3.,0.) circle (0.5cm);
            \draw (-3,0) node[anchor=center] {$A$};
            \draw (-1,0) node[anchor=center] {$B$};
            \draw (1,0) node[anchor=center] {$C$};
            \draw (3,0) node[anchor=center] {$D$};
            \draw [<->,>=stealth,shorten <=1.pt,shorten >=1.pt,line width=1.pt] (-2.5,0.)-- (-1.5,0.);
            \draw [->,>=stealth,shorten <=1.pt,shorten >=1.pt,line width=1.pt] (-0.5,0.)-- (0.5,0.);
            \draw [->,>=stealth,shorten <=1.pt,shorten >=1.pt,line width=1.pt] (1.5,0.)-- (2.5,0.);
            \draw [<->,>=stealth,shorten <=1.pt,shorten >=1.pt,shift={(3.5,0.)},line width=1.pt]  plot[domain=-2.0943951023931957:2.0943951023931957,variable=\t]({1.*0.5*cos(\t r)+0.*0.5*sin(\t r)},{0.*0.5*cos(\t r)+1.*0.5*sin(\t r)});
        \end{tikzpicture}
    }

    Relation \(R^{TC}\)

    \resizebox{21em}{12em}{
        \begin{tikzpicture}[line cap=round,line join=round,>=triangle 45,x=1.0cm,y=1.0cm]
            \clip(-4.5,-3.25) rectangle (4.5,2.);
            \draw [line width=1.pt] (-3.,0.) circle (0.5cm);
            \draw [line width=1.pt] (-1.,0.) circle (0.5cm);
            \draw [line width=1.pt] (1.,0.) circle (0.5cm);
            \draw [line width=1.pt] (3.,0.) circle (0.5cm);
            \draw (-3,0) node[anchor=center] {$A$};
            \draw (-1,0) node[anchor=center] {$B$};
            \draw (1,0) node[anchor=center] {$C$};
            \draw (3,0) node[anchor=center] {$D$};
            \draw [<->,>=stealth,shorten <=1.pt,shorten >=1.pt,line width=1.pt] (-2.5,0.)-- (-1.5,0.);
            \draw [->,>=stealth,shorten <=1.pt,shorten >=1.pt,line width=1.pt] (-0.5,0.)-- (0.5,0.);
            \draw [->,>=stealth,shorten <=1.pt,shorten >=1.pt,line width=1.pt] (1.5,0.)-- (2.5,0.);
            \draw [blue,<-,>=stealth,shorten <=3.pt,shorten >=3.pt,shift={(-1.,-0.5)},line width=1.pt]  plot[domain=0.4636476090008061:2.677945044588987,variable=\t]({1.*2.23606797749979*cos(\t r)+0.*2.23606797749979*sin(\t r)},{0.*2.23606797749979*cos(\t r)+1.*2.23606797749979*sin(\t r)});
            \draw [blue,->,>=stealth,shorten <=3.pt,shorten >=3.pt,shift={(1.,0.5)},line width=1.pt]  plot[domain=3.6052402625905993:5.81953769817878,variable=\t]({1.*2.23606797749979*cos(\t r)+0.*2.23606797749979*sin(\t r)},{0.*2.23606797749979*cos(\t r)+1.*2.23606797749979*sin(\t r)});
            \draw [blue,->,>=stealth,shorten <=3.pt,shorten >=3.pt,shift={(0.,0.)},line width=1.pt]  plot[domain=3.30674133100442:6.118036629764959,variable=\t]({1.*3.0413812651491092*cos(\t r)+0.*3.0413812651491092*sin(\t r)},{0.*3.0413812651491092*cos(\t r)+1.*3.0413812651491092*sin(\t r)});
            \draw [blue,<->,>=stealth,shorten <=1.pt,shorten >=1.pt,shift={(-3.5,0.)},line width=1.pt]  plot[domain=1.0471975511965976:5.235987755982989,variable=\t]({1.*0.5*cos(\t r)+0.*0.5*sin(\t r)},{0.*0.5*cos(\t r)+1.*0.5*sin(\t r)});
            \draw [<->,>=stealth,shorten <=1.pt,shorten >=1.pt,shift={(3.5,0.)},line width=1.pt]  plot[domain=-2.0943951023931957:2.0943951023931957,variable=\t]({1.*0.5*cos(\t r)+0.*0.5*sin(\t r)},{0.*0.5*cos(\t r)+1.*0.5*sin(\t r)});
            \draw [blue,<->,>=stealth,shorten <=1.pt,shorten >=1.pt,shift={(-1.,0.5)},line width=1.pt]  plot[domain=-0.5235987755982991:3.6651914291880923,variable=\t]({1.*0.5*cos(\t r)+0.*0.5*sin(\t r)},{0.*0.5*cos(\t r)+1.*0.5*sin(\t r)});
        \end{tikzpicture}
    }
\end{figure}

Closure can be defined for other properties:

\begin{definition}
    Given \(X\in\set{\text{``transitive'', ``symmetric", ``reflexive", ``reflexive and transitive''}}\), the \(X\) closure \index{Set Theory!Closure on Relations}of binary relation \(R\) 
    on set \(S\) is the smallest relation on \(S\) that includes \(R\) and has property \(X\).
\end{definition}

Transitivity is interesting as it represents the idea of reachability. 

\bigskip
\begin{definition}
    The \emph{transitive reduction}\index{Set Theory!Transitive Reduction} of a binary relation \(R\) on set \(S\) is the smallest relation \(R^*\) on \(S\) where \((R^*)^{TR}=R^{TR}\).
\end{definition}

Intuitively, transitive reduction is the action of removing elements of \(S\times S\) from the relation such that the transitive closure would still remain the same. 

\begin{figure}[H]
    \centering
    \textbf{Example of Transitive Reduction}

    Relation \(R\)

    \resizebox{20em}{5em}{
        \begin{tikzpicture}[line cap=round,line join=round,>=triangle 45,x=1.0cm,y=1.0cm]
            \clip(-3.,-2.) rectangle (13.,2.);
            \draw [line width=2.pt] (0.,0.) circle (1.cm);
            \draw [line width=2.pt] (10.,0.) circle (1.cm);
            \draw [<->,shorten <=3.pt,shorten >=3.pt,shift={(11.,0.)},line width=2.pt]  plot[domain=-2.094395102393194:2.094395102393194,variable=\t]({1.*1.*cos(\t r)+0.*1.*sin(\t r)},{0.*1.*cos(\t r)+1.*1.*sin(\t r)});
            \draw [<->,shorten <=3.pt,shorten >=3.pt,shift={(-1.,0.)},line width=2.pt]  plot[domain=1.0471975511965976:5.235987755982989,variable=\t]({1.*1.*cos(\t r)+0.*1.*sin(\t r)},{0.*1.*cos(\t r)+1.*1.*sin(\t r)});
            \draw [<->,shorten <=3.pt,shorten >=3.pt,line width=2.pt] (1.,0.)-- (9.,0.);
            \draw (9.6,0.4) node[anchor=north west] {\LARGE$B$};
            \draw (-0.3,0.4) node[anchor=north west] {\LARGE$D$};
        \end{tikzpicture}
    }

    Relation \(R^*\)

    \resizebox{20em}{5em}{
        \begin{tikzpicture}[line cap=round,line join=round,>=triangle 45,x=1.0cm,y=1.0cm]
            \clip(-3.,-2.) rectangle (13.,2.);
            \draw [line width=2.pt] (0.,0.) circle (1.cm);
            \draw [line width=2.pt] (10.,0.) circle (1.cm);
            \draw [red,<->,shorten <=3.pt,shorten >=3.pt,shift={(11.,0.)},line width=2.pt]  plot[domain=-2.094395102393194:2.094395102393194,variable=\t]({1.*1.*cos(\t r)+0.*1.*sin(\t r)},{0.*1.*cos(\t r)+1.*1.*sin(\t r)});
            \draw [red,<->,shorten <=3.pt,shorten >=3.pt,shift={(-1.,0.)},line width=2.pt]  plot[domain=1.0471975511965976:5.235987755982989,variable=\t]({1.*1.*cos(\t r)+0.*1.*sin(\t r)},{0.*1.*cos(\t r)+1.*1.*sin(\t r)});
            \draw [<->,shorten <=3.pt,shorten >=3.pt,line width=2.pt] (1.,0.)-- (9.,0.);
            \draw (9.6,0.4) node[anchor=north west] {\LARGE$B$};
            \draw (-0.3,0.4) node[anchor=north west] {\LARGE$D$};
        \end{tikzpicture}
    }

    The red arrows denote removed elements of relation \(R\).
    \newline In this example, \(R\) is transitive already. 
\end{figure}
\begin{figure}[H]
    \centering
    \textbf{Example of Transitive Reduction}

    Relation \(R\)

    \resizebox{21em}{12em}{
        \begin{tikzpicture}[line cap=round,line join=round,>=triangle 45,x=1.0cm,y=1.0cm]
            \clip(-4.5,-3.25) rectangle (4.5,2.);
            \draw [line width=1.pt] (-3.,0.) circle (0.5cm);
            \draw [line width=1.pt] (-1.,0.) circle (0.5cm);
            \draw [line width=1.pt] (1.,0.) circle (0.5cm);
            \draw [line width=1.pt] (3.,0.) circle (0.5cm);
            \draw (-3,0) node[anchor=center] {$A$};
            \draw (-1,0) node[anchor=center] {$B$};
            \draw (1,0) node[anchor=center] {$C$};
            \draw (3,0) node[anchor=center] {$D$};
            \draw [<->,>=stealth,shorten <=1.pt,shorten >=1.pt,line width=1.pt] (-2.5,0.)-- (-1.5,0.);
            \draw [->,>=stealth,shorten <=1.pt,shorten >=1.pt,line width=1.pt] (-0.5,0.)-- (0.5,0.);
            \draw [->,>=stealth,shorten <=1.pt,shorten >=1.pt,line width=1.pt] (1.5,0.)-- (2.5,0.);
            \draw [<-,>=stealth,shorten <=3.pt,shorten >=3.pt,shift={(-1.,-0.5)},line width=1.pt]  plot[domain=0.4636476090008061:2.677945044588987,variable=\t]({1.*2.23606797749979*cos(\t r)+0.*2.23606797749979*sin(\t r)},{0.*2.23606797749979*cos(\t r)+1.*2.23606797749979*sin(\t r)});
            \draw [->,>=stealth,shorten <=3.pt,shorten >=3.pt,shift={(1.,0.5)},line width=1.pt]  plot[domain=3.6052402625905993:5.81953769817878,variable=\t]({1.*2.23606797749979*cos(\t r)+0.*2.23606797749979*sin(\t r)},{0.*2.23606797749979*cos(\t r)+1.*2.23606797749979*sin(\t r)});
            \draw [<->,>=stealth,shorten <=1.pt,shorten >=1.pt,shift={(-3.5,0.)},line width=1.pt]  plot[domain=1.0471975511965976:5.235987755982989,variable=\t]({1.*0.5*cos(\t r)+0.*0.5*sin(\t r)},{0.*0.5*cos(\t r)+1.*0.5*sin(\t r)});
            \draw [<->,>=stealth,shorten <=1.pt,shorten >=1.pt,shift={(3.5,0.)},line width=1.pt]  plot[domain=-2.0943951023931957:2.0943951023931957,variable=\t]({1.*0.5*cos(\t r)+0.*0.5*sin(\t r)},{0.*0.5*cos(\t r)+1.*0.5*sin(\t r)});
        \end{tikzpicture}
    }

    Relation \(R^*\)

    \resizebox{21em}{12em}{
        \begin{tikzpicture}[line cap=round,line join=round,>=triangle 45,x=1.0cm,y=1.0cm]
            \clip(-4.5,-3.25) rectangle (4.5,2.);
            \draw [line width=1.pt] (-3.,0.) circle (0.5cm);
            \draw [line width=1.pt] (-1.,0.) circle (0.5cm);
            \draw [line width=1.pt] (1.,0.) circle (0.5cm);
            \draw [line width=1.pt] (3.,0.) circle (0.5cm);
            \draw (-3,0) node[anchor=center] {$A$};
            \draw (-1,0) node[anchor=center] {$B$};
            \draw (1,0) node[anchor=center] {$C$};
            \draw (3,0) node[anchor=center] {$D$};
            \draw [<->,>=stealth,shorten <=1.pt,shorten >=1.pt,line width=1.pt] (-2.5,0.)-- (-1.5,0.);
            \draw [->,>=stealth,shorten <=1.pt,shorten >=1.pt,line width=1.pt] (-0.5,0.)-- (0.5,0.);
            \draw [->,>=stealth,shorten <=1.pt,shorten >=1.pt,line width=1.pt] (1.5,0.)-- (2.5,0.);
            \draw [red,<-,>=stealth,shorten <=3.pt,shorten >=3.pt,shift={(-1.,-0.5)},line width=1.pt]  plot[domain=0.4636476090008061:2.677945044588987,variable=\t]({1.*2.23606797749979*cos(\t r)+0.*2.23606797749979*sin(\t r)},{0.*2.23606797749979*cos(\t r)+1.*2.23606797749979*sin(\t r)});
            \draw [red,->,>=stealth,shorten <=3.pt,shorten >=3.pt,shift={(1.,0.5)},line width=1.pt]  plot[domain=3.6052402625905993:5.81953769817878,variable=\t]({1.*2.23606797749979*cos(\t r)+0.*2.23606797749979*sin(\t r)},{0.*2.23606797749979*cos(\t r)+1.*2.23606797749979*sin(\t r)});
            \draw [red,<->,>=stealth,shorten <=1.pt,shorten >=1.pt,shift={(-3.5,0.)},line width=1.pt]  plot[domain=1.0471975511965976:5.235987755982989,variable=\t]({1.*0.5*cos(\t r)+0.*0.5*sin(\t r)},{0.*0.5*cos(\t r)+1.*0.5*sin(\t r)});
            \draw [<->,>=stealth,shorten <=1.pt,shorten >=1.pt,shift={(3.5,0.)},line width=1.pt]  plot[domain=-2.0943951023931957:2.0943951023931957,variable=\t]({1.*0.5*cos(\t r)+0.*0.5*sin(\t r)},{0.*0.5*cos(\t r)+1.*0.5*sin(\t r)});
        \end{tikzpicture}
    }

    Note that \(R\) is not transitive.
\end{figure}

\section{Algorithms for Computing Transitive Closure}

There are some algorithmic methods of calculating the transitive closure.

\subsection{The Naive Approach}
Given relation \(R\) on set \(S\), transitivity states that \(\forall\:a,b,c\in S\:(a,b)\in R\land (b,c)\in R\Rightarrow(a,c)\in R\). First let \(R^{TC}=R\).
Then, the algorithm involves looping through all the elements of \(S\) for elements \(a,b,c\) and determining if \((a,b)\in R^{TC}\) and \((b,c)\in R^{TC}\). 
If so, then the new relation will have \((a,c)\in R^{TC}\). However, this new relation will cause new transitive relations to be required in \(R^{TC}\). 
The algorithms continues to loop through \(S\). The result will be that \(R^{TC}\) is the transitive closure of \(R\). This process will take \(\mathcal{O}(n^4)\).

\subsection{Depth-First Search Approach}
A depth-first search approach (typically implemented through recursion) essentially involves determining all the nodes that are "reachable" from a node. A node
is reachable, if it is either in the relation or there is a ``path'' to the node through transitivity. This process will take \(\mathcal{O}(n^3)\).

\subsection{Adjacency Matrix Approach}
The relation \(R\) on set \(S\) is represented graphically. This graph can then be interpreted as an \emph{adjacency matrix}.\footnote{More information on the adjacency
matrix can be found the following pages of these textbooks:\\Alan Tucker's ``Introduction to Linear Algebra: Models, Methods, and Theory" Section 2.3 p.99\\
Gilbert Strang's ``Introduction to Linear Algebra" Section 2.4 p.76} The adjacency matrix will be a \(|S|\times|S|\) matrix with \(1\) in entries corresponding to an 
element of \(x\in S\times S\) where \(x\in R\), and \(0\) elsewhere. The \(1\)s represent that there is one ``path'' from the two nodes that can be taken in one step. 
Take the following example:
\[S=\set{A,B,C,D}\]
\[R\subseteq S\times S\]
\[R=\set{(A,B),(B,A),(B,C),(C,D),(D,D)}\]
The relation is represented by the following graph:
\begin{figure}[H]
    \centering
    \textbf{Relation \(R\) on set \(S\)}

    \resizebox{21em}{5em}{
        \begin{tikzpicture}[line cap=round,line join=round,>=triangle 45,x=1.0cm,y=1.0cm]
            \clip(-4.,-1.) rectangle (4.5,1.);
            \draw [line width=1.pt] (-3.,0.) circle (0.5cm);
            \draw [line width=1.pt] (-1.,0.) circle (0.5cm);
            \draw [line width=1.pt] (1.,0.) circle (0.5cm);
            \draw [line width=1.pt] (3.,0.) circle (0.5cm);
            \draw (-3,0) node[anchor=center] {$A$};
            \draw (-1,0) node[anchor=center] {$B$};
            \draw (1,0) node[anchor=center] {$C$};
            \draw (3,0) node[anchor=center] {$D$};
            \draw [<->,>=stealth,shorten <=1.pt,shorten >=1.pt,line width=1.pt] (-2.5,0.)-- (-1.5,0.);
            \draw [->,>=stealth,shorten <=1.pt,shorten >=1.pt,line width=1.pt] (-0.5,0.)-- (0.5,0.);
            \draw [->,>=stealth,shorten <=1.pt,shorten >=1.pt,line width=1.pt] (1.5,0.)-- (2.5,0.);
            \draw [<->,>=stealth,shorten <=1.pt,shorten >=1.pt,shift={(3.5,0.)},line width=1.pt]  plot[domain=-2.0943951023931957:2.0943951023931957,variable=\t]({1.*0.5*cos(\t r)+0.*0.5*sin(\t r)},{0.*0.5*cos(\t r)+1.*0.5*sin(\t r)});
        \end{tikzpicture}
    }
\end{figure}
The adjacency matrix \(A\) of relation \(R\) is:
\[A=\begin{bmatrix}
    0 & 1 & 0 & 0 \\
    1 & 0 & 1 & 0 \\
    0 & 0 & 0 & 1 \\
    0 & 0 & 0 & 1
\end{bmatrix}\] 
However, the matrix needs to be modified. Currently, the matrices represent the possible paths between the
two nodes, but this always requires a traversal through an edge of the graph (an element of the relation).
The identity matrix \(I\) will be added to \(A\). This identity matrix will encode the new action of ``staying
in place.'' This will add a new ``path'' for elements to remain in place. 
\begin{align*}
    M&=A+I \\ 
     &=\begin{bmatrix}
        1 & 1 & 0 & 0 \\
        1 & 1 & 1 & 0 \\
        0 & 0 & 1 & 1 \\
        0 & 0 & 1 & 2
     \end{bmatrix}
\end{align*}
I will denote entry of matrix \(M\) on row \(i\) and column \(j\) as \(M_{i,j}\). Some things to note about this 
new matrix:
\begin{itemize}
    \item Note that \((A,A)\not\in R\), but \(M_{1,1}=1\). This \(1\) represents how there is one way to reach \(A\)
    from \(A\), i.e.\ staying at \(A\).
    \item Also note that \(M_{4,4}=2\). This represents how there are now two ways to reach \(D\) from \(D\). The first
    is to follow the path represented by \((D,D)\in R\). The other way is simply not from that node at all.
\end{itemize}
The benefits of the adjacency matrix is that its powers encode the number of paths between two nodes. For example,
\(M^2\) will encode the number of ways you can travel from two nodes in two steps (I am using step synonymously with
an action. This includes taking a path defined in the relation or staying in place). \(M^3\) encodes the number of ways
you can travel between two nodes in three steps. Generally, \(M^n\) encodes the number of ways you can travel between
two nodes in \(n\) steps. These paths are important as transitivity would imply there is a direct path between the two
nodes. Thus, \(M^n\) will fill in the entries that the transitive closure will have.

Because there are \(4\) nodes in the example, \(M^4\) will be adequate to compute enough information to compute the
transitive closure matrix. Generally, \(M^{|S|}\) will be enough. Extra matrix multiplications will not affect the
general makeup of the computed matrix. I am going to be using MATLAB to compute the matrix:
\begin{lstlisting}[language=Matlab]
A=[0,1,0,0;
   1,0,1,0;
   0,0,0,1;
   0,0,0,1];
M=A+eye(4);   
disp(M^4); 
\end{lstlisting}
This yields the following matrix:
\[M^4=\begin{bmatrix}
    8 & 8 & 7 &  5 \\
    8 & 8 & 8 & 12 \\
    0 & 0 & 1 & 15 \\
    0 & 0 & 0 & 16
\end{bmatrix}\]
The most important part to note is that the diagonal count an extra path. This path originates from when we added the identity
matrix to \(A\). For example, \(M^4_{3,3}=1\) means there is one path from node \(C\) to itself that is taken in four steps.
This path is simply doing nothing for four steps. This path should not be represented in adjacency matrix for transitive closure.
Therefore, this possible path needs to be removed from \(M^4\). This is simply done by subtracting the identity matrix
from the result, i.e.\ \(M^4-I\). 
\[M^4-I=\begin{bmatrix}
    7 & 8 & 7 &  5 \\
    8 & 7 & 8 & 12 \\
    0 & 0 & 0 & 15 \\
    0 & 0 & 0 & 15
\end{bmatrix}\]
Therefore, the transitive closure can be derived from this adjacency matrix:
\[R^{TC}=\set{(a,a),(a,b),(a,c),(a,d),(b,a),(b,b),(b,c),(b,d),(c,d),(d,d)}\]

\begin{figure}[H]
    \centering
    \textbf{Relation \(R^{TC}\)}

    \resizebox{21em}{12em}{
        \begin{tikzpicture}[line cap=round,line join=round,>=triangle 45,x=1.0cm,y=1.0cm]
            \clip(-4.5,-3.25) rectangle (4.5,2.);
            \draw [line width=1.pt] (-3.,0.) circle (0.5cm);
            \draw [line width=1.pt] (-1.,0.) circle (0.5cm);
            \draw [line width=1.pt] (1.,0.) circle (0.5cm);
            \draw [line width=1.pt] (3.,0.) circle (0.5cm);
            \draw (-3,0) node[anchor=center] {$A$};
            \draw (-1,0) node[anchor=center] {$B$};
            \draw (1,0) node[anchor=center] {$C$};
            \draw (3,0) node[anchor=center] {$D$};
            \draw [<->,>=stealth,shorten <=1.pt,shorten >=1.pt,line width=1.pt] (-2.5,0.)-- (-1.5,0.);
            \draw [->,>=stealth,shorten <=1.pt,shorten >=1.pt,line width=1.pt] (-0.5,0.)-- (0.5,0.);
            \draw [->,>=stealth,shorten <=1.pt,shorten >=1.pt,line width=1.pt] (1.5,0.)-- (2.5,0.);
            \draw [<-,>=stealth,shorten <=3.pt,shorten >=3.pt,shift={(-1.,-0.5)},line width=1.pt]  plot[domain=0.4636476090008061:2.677945044588987,variable=\t]({1.*2.23606797749979*cos(\t r)+0.*2.23606797749979*sin(\t r)},{0.*2.23606797749979*cos(\t r)+1.*2.23606797749979*sin(\t r)});
            \draw [->,>=stealth,shorten <=3.pt,shorten >=3.pt,shift={(1.,0.5)},line width=1.pt]  plot[domain=3.6052402625905993:5.81953769817878,variable=\t]({1.*2.23606797749979*cos(\t r)+0.*2.23606797749979*sin(\t r)},{0.*2.23606797749979*cos(\t r)+1.*2.23606797749979*sin(\t r)});
            \draw [->,>=stealth,shorten <=3.pt,shorten >=3.pt,shift={(0.,0.)},line width=1.pt]  plot[domain=3.30674133100442:6.118036629764959,variable=\t]({1.*3.0413812651491092*cos(\t r)+0.*3.0413812651491092*sin(\t r)},{0.*3.0413812651491092*cos(\t r)+1.*3.0413812651491092*sin(\t r)});
            \draw [<->,>=stealth,shorten <=1.pt,shorten >=1.pt,shift={(-3.5,0.)},line width=1.pt]  plot[domain=1.0471975511965976:5.235987755982989,variable=\t]({1.*0.5*cos(\t r)+0.*0.5*sin(\t r)},{0.*0.5*cos(\t r)+1.*0.5*sin(\t r)});
            \draw [<->,>=stealth,shorten <=1.pt,shorten >=1.pt,shift={(3.5,0.)},line width=1.pt]  plot[domain=-2.0943951023931957:2.0943951023931957,variable=\t]({1.*0.5*cos(\t r)+0.*0.5*sin(\t r)},{0.*0.5*cos(\t r)+1.*0.5*sin(\t r)});
            \draw [<->,>=stealth,shorten <=1.pt,shorten >=1.pt,shift={(-1.,0.5)},line width=1.pt]  plot[domain=-0.5235987755982991:3.6651914291880923,variable=\t]({1.*0.5*cos(\t r)+0.*0.5*sin(\t r)},{0.*0.5*cos(\t r)+1.*0.5*sin(\t r)});
        \end{tikzpicture}
    }
\end{figure}
In conclusion, the algorithm works as follows; given relation \(R\) on set \(S\), let \(A\) be the adjacency matrix of
\(R\). The transitive closure of \(R\) is represented by the adjacency matrix \((A+I)^{|S|}-I\).

Matrix multiplication typically done in \(\mathcal{O}(n^3)\). There are algorithms that exist that can perform matrix multiplication
with a lower time complexity like the Strassen Algorithm which has approximately \(\mathcal{O}(n^{2.81})\). However, since it is 
asymptomatic behavior being described, the Strassen Algorithm slower than the original approach when it pertains to smaller matrices.
\footnote{If you are interested in more pragmatic algorithms for matrix multiplication, see https://en.algorithmica.org/hpc/algorithms/matmul/}
Matrix multiplication is clearly an expensive operation. Therefore, it is important to reduce the number of required matrix multiplications.

To reduce the number of required matrix multiplication, the product \(M^{|S|}\) can be computed in the same way as it is done in integer
binary exponentiation. Therefore, this process takes \(\mathcal{O}(\log_2n)\). In conclusion, this algorithm will have 
\(\mathcal{O}(\log n\epsilon)\) where \(\epsilon\) is the time complexity for the matrix multiplication operation. 

As an aside, \((A+I)^{|S|}-I\) can be expanded.
\[(A+I)^{|S|}-I=\sum_{i=1}^{|S|}A^i\]
Intuitively, this equation can be viewed as the sum of all the number of paths between two nodes that takes one to \(|S|\) steps.
If such an indirect path between two nodes exist, then the transitive closure will have a direct path between the two nodes. Therefore,
\(\sum_{i=1}^{|S|}A^i\) represents the adjacency matrix for the transitive closure.