\lecturechapter{Written by Jeffrey J.}

\section{Introduction to Functions}
\index{Functions}

\subsection{Basic Definition}

\begin{definition}
    A \emph{function} \(f:\:S\rightarrow T\) is a relation \(f\subseteq S\times T\) such that \(\forall\:s\in S\:\exists!\: t\in T\ (s,t)\in f\).
\end{definition}

Notation-wise, \((s,t)\in f\) can be written as \(f(s)=t\). A function can be represented graphically:
\begin{figure}[H]
    \centering
    \textbf{Example of a Function}

    \resizebox{16em}{12em}{
        \begin{tikzpicture}[line cap=round,line join=round,>=triangle 45,x=1.0cm,y=1.0cm]
            \clip(-3.5,-1.5) rectangle (3.5,3.5);
            \draw [rotate around={90.:(-2.,0.5)},line width=2.pt] (-2.,0.5) ellipse (1.825140769936443cm and 1.039778260055571cm);
            \draw [rotate around={90.:(2.,0.5)},line width=2.pt] (2.,0.5) ellipse (1.825140769936443cm and 1.039778260055571cm);
            \draw (-2,3) node[anchor=center] {\LARGE$S$};
            \draw (2,3) node[anchor=center] {\LARGE$T$};
            \draw [->,>=stealth,shorten <=5.pt,shorten >=5.pt, line width=2.pt] (-2.,2.)-- (2.,2.);
            \draw [->,>=stealth,shorten <=5.pt,shorten >=5.pt, line width=2.pt] (-2.,1.)-- (2.,2.);
            \draw [->,>=stealth,shorten <=5.pt,shorten >=5.pt, line width=2.pt] (-2.,0.)-- (2.,0.);
            \draw [->,>=stealth,shorten <=5.pt,shorten >=5.pt, line width=2.pt] (-2.,-1.)-- (2.,-1.);
            \begin{scriptsize}
            \draw [fill=black] (-2.,2.) circle (3pt);
            \draw [fill=black] (-2.,1.) circle (3pt);
            \draw [fill=black] (-2.,0.) circle (3pt);
            \draw [fill=black] (-2.,-1.) circle (3pt);
            \draw [fill=black] (2.,2.) circle (3pt);
            \draw [fill=black] (2.,1.) circle (3pt);
            \draw [fill=black] (2.,0.) circle (3pt);
            \draw [fill=black] (2.,-1.) circle (3pt);
            \end{scriptsize}
        \end{tikzpicture}
    }
\end{figure}

\begin{definition}
    Given function \(f:\:S\rightarrow T\), \(S\) is the \emph{domain}. \(T\) is the \emph{codomain}.
\end{definition}

\bigskip
\begin{definition}
    Given function \(f:\: S\rightarrow T\), the \emph{range} is the set \(\setbuild{f(x)\in T}{x\in S}\).
    \footnote{The range is also called the \emph{image} of \(f\). The domain is thus the \emph{pre-image}.}
\end{definition}

Clearly, the range \(R\) of function \(f:\:S\rightarrow T\) is the subset of the codomain of the function.

\subsection{Basic Properties of Functions}

\begin{definition}
    A function \(f:\:S\rightarrow T\) is \emph{injective}\index{Functions!Injection}, or \emph{one-to-one}, if \(\forall\:a,b\in S\:(a=b)\Rightarrow(f(a)=f(b))\).
\end{definition}

Intuitively, this means that there is at most one input for each output. Thus, this is a \emph{one-to-one} mapping. An example:
\begin{figure}[H]
    \centering
    \textbf{Example of Injection}

    \resizebox*{16em}{12em}{
        \begin{tikzpicture}[line cap=round,line join=round,>=triangle 45,x=1.0cm,y=1.0cm]
            \clip(-3.5,-1.5) rectangle (3.5,3.5);
            \draw [rotate around={90.:(-2.,0.5)},line width=2.pt] (-2.,0.5) ellipse (1.825140769936443cm and 1.039778260055571cm);
            \draw [rotate around={90.:(2.,0.5)},line width=2.pt] (2.,0.5) ellipse (1.825140769936443cm and 1.039778260055571cm);
            \draw (-2,3) node[anchor=center] {\LARGE$S$};
            \draw (2,3) node[anchor=center] {\LARGE$T$};
            \draw [->,>=stealth,shorten <=5.pt,shorten >=5.pt, line width=2.pt] (-2.,2.)-- (2.,2.);
            \draw [->,>=stealth,shorten <=5.pt,shorten >=5.pt, line width=2.pt] (-2.,1.)-- (2.,1.);
            \draw [->,>=stealth,shorten <=5.pt,shorten >=5.pt, line width=2.pt] (-2.,0.)-- (2.,0.);
            \begin{scriptsize}
            \draw [fill=black] (-2.,2.) circle (2.5pt);
            \draw [fill=black] (-2.,1.) circle (2.5pt);
            \draw [fill=black] (-2.,0.) circle (2.5pt);
            \draw [fill=black] (2.,2.) circle (2.5pt);
            \draw [fill=black] (2.,1.) circle (2.5pt);
            \draw [fill=black] (2.,0.) circle (2.5pt);
            \draw [fill=black] (2.,-1.) circle (2.5pt);
            \end{scriptsize}
        \end{tikzpicture}
    }
\end{figure}

\bigskip
\begin{proposition}
    If function \(f:\: S\rightarrow T\) is injective, then \(|S| \le |T|\).
\end{proposition}
\begin{proof}
    By definition of injection, every element of \(S\) has a one-to-one correspondence to an element in \(T\). More precisely,
    \(|S|=|\text{im}(f)|\) where \(\text{im}(f)\) is the image of \(f\), i.e.\ \(\text{im}(f)=\setbuild{f(x)\in T}{x\in S}\).
    \(\text{im}(f)\subseteq T\) as all elements of \(\text{im}(f)\) is an element of \(T\) by definition of the function.
    In conclusion, \(|S|=|\text{im}(f)|\le|T|\).
\end{proof}

\bigskip
\begin{definition}
    A function \(f:\:S\rightarrow T\) is \emph{surjective}\index{Functions!Surjection}, or \emph{onto}, if \(\forall\:t\in T\existss s\in S\ (s,t)\in f\).
\end{definition}

Intuitively, this means that every value in the codomain has an input in the domain that yields that value. In this sense, the domain
completely maps \emph{onto} the codomain. An example:
\begin{figure}[H]
    \centering
    \textbf{Example of Surjection}

    \resizebox*{16em}{12em}{
        \begin{tikzpicture}[line cap=round,line join=round,>=triangle 45,x=1.0cm,y=1.0cm]
            \clip(-3.5,-1.5) rectangle (3.5,3.5);
            \draw [rotate around={90.:(-2.,0.5)},line width=2.pt] (-2.,0.5) ellipse (1.825140769936443cm and 1.039778260055571cm);
            \draw [rotate around={90.:(2.,0.5)},line width=2.pt] (2.,0.5) ellipse (1.825140769936443cm and 1.039778260055571cm);
            \draw (-2,3) node[anchor=center] {\LARGE$S$};
            \draw (2,3) node[anchor=center] {\LARGE$T$};
            \draw [->,>=stealth,shorten <=5.pt,shorten >=5.pt,line width=2.pt] (-2.,2.)-- (2.,2.);
            \draw [->,>=stealth,shorten <=5.pt,shorten >=5.pt,line width=2.pt] (-2.,1.)-- (2.,1.);
            \draw [->,>=stealth,shorten <=5.pt,shorten >=5.pt,line width=2.pt] (-2.,0.)-- (2.,0.);
            \draw [->,>=stealth,shorten <=5.pt,shorten >=5.pt,line width=2.pt] (-2.,-1.)-- (2.,0.);
            \begin{scriptsize}
            \draw [fill=black] (-2.,2.) circle (2.5pt);
            \draw [fill=black] (-2.,1.) circle (2.5pt);
            \draw [fill=black] (-2.,0.) circle (2.5pt);
            \draw [fill=black] (-2.,-1.) circle (2.5pt);
            \draw [fill=black] (2.,2.) circle (2.5pt);
            \draw [fill=black] (2.,1.) circle (2.5pt);
            \draw [fill=black] (2.,0.) circle (2.5pt);
            \end{scriptsize}
        \end{tikzpicture}
    }
\end{figure}

\bigskip
\begin{proposition}
    If function \(f:\: S\rightarrow T\) is surjective, then \(|S|\ge|T|\).
\end{proposition}
\begin{proof}
    By definition of surjection, every element in the codomain has at least one element of the domain mapping to it. Assume \(|S|< |T|\).
    If there is a one-to-one mapping or many-to-one, then this means that exists element(s) of \(T\) which does not have mappings from
    \(S\) to \(T\). This contradicts the definition of surjection meaning \(|S|\ge|T|\). 
\end{proof}

\bigskip
\begin{definition}
    A function \(f:\: S\rightarrow T\) is \emph{bijective}\index{Functions!Bijection} if it is injective and surjective. 
\end{definition}

An important property of bijections is that they have a left-inverse and right-inverse. Given function, then there exists 
\(f^{-1}\) such that \(f\circ f^{-1}=1_S\) and \(f^{-1}\circ f=1_T\) where \(1\) represents the identity bijection. 
The identity bijection of any set \(S\) is the following:
\begin{align*}
    1_S:\quad S&\longrightarrow S \\
            s&\longrightarrow s
\end{align*}
This is not true for surjections and injections.

For injection \(f:\:S\rightarrow T\), inverse function \(f^{-1}: T\rightarrow S\) does
not necessarily exist because there may exist elements of \(T\) which does not have an element of \(S\) mapping to it. Therefore, \(f^{-1}\)
is not a function. However, an function \(f^{-1}:\:\text{im}(f)\rightarrow S\) can be defined where all the corresponding elements of the range
is mapped to its input.

For surjections \(f:\:S\rightarrow T\). the inverse function also does not exist because is is possible for many elements of the domain map to the
same element of the range. As a result, the inverse will not be a function.

An example of a bijection:
\begin{figure}[H]
    \centering
    \textbf{Example of Bijection}

    \resizebox{16em}{12em}{
        \begin{tikzpicture}[line cap=round,line join=round,>=triangle 45,x=1.0cm,y=1.0cm]
            \clip(-3.5,-1.5) rectangle (3.5,3.5);
            \draw [rotate around={90.:(-2.,0.5)},line width=2.pt] (-2.,0.5) ellipse (1.825140769936443cm and 1.039778260055571cm);
            \draw [rotate around={90.:(2.,0.5)},line width=2.pt] (2.,0.5) ellipse (1.825140769936443cm and 1.039778260055571cm);
            \draw (-2,3) node[anchor=center] {\LARGE$S$};
            \draw (2,3) node[anchor=center] {\LARGE$T$};
            \draw [->,>=stealth,shorten <=5.pt,shorten >=5.pt,line width=2.pt] (-2.,2.)-- (2.,2.);
            \draw [->,>=stealth,shorten <=5.pt,shorten >=5.pt,line width=2.pt] (-2.,1.)-- (2.,1.);
            \draw [->,>=stealth,shorten <=5.pt,shorten >=5.pt,line width=2.pt] (-2.,0.)-- (2.,0.);
            \draw [->,>=stealth,shorten <=5.pt,shorten >=5.pt,line width=2.pt] (-2.,-1.)-- (2.,-1.);
            \begin{scriptsize}
            \draw [fill=black] (-2.,2.) circle (2.5pt);
            \draw [fill=black] (-2.,1.) circle (2.5pt);
            \draw [fill=black] (-2.,0.) circle (2.5pt);
            \draw [fill=black] (-2.,-1.) circle (2.5pt);
            \draw [fill=black] (2.,2.) circle (2.5pt);
            \draw [fill=black] (2.,1.) circle (2.5pt);
            \draw [fill=black] (2.,0.) circle (2.5pt);
            \draw [fill=black] (2.,-1.) circle (2.5pt);
            \end{scriptsize}
        \end{tikzpicture}
    }
\end{figure}

\bigskip
\begin{proposition}
    If a function \(f:\:S\rightarrow T\) is a bijection, then \(|S|=|T|\).
\end{proposition}
\begin{proof}
    Function \(f\) is bijective meaning it is both injective and surjective. By proposition \(2.1\), \(|S|\le|T|\) as
    the function is injective. Likewise, by proposition \(2.2\), \(|S|\ge|T|\) as the function is surjective. Therefore,
    \(|S|=|T|\) as both of these statements is true which is only possible if the order of the sets are equal.
\end{proof}

\section{Counting and Infinity}
\index{Counting}

\subsection{Basic Definitions}

\begin{definition}
    A set \(A\) is \emph{finite}\index{Counting!finite} if there exists a bijection between \(A\) and \(\set{1,2,\dots,n}\) for some \(n\ge0\).
    Additionally, \(|A|=n\).
\end{definition}

\bigskip
\begin{definition}
    A set \(A\) is \emph{infinite}\index{Counting!infinite} if it is not finite.
\end{definition}

\bigskip
\begin{definition}
    A set \(A\) is \emph{countably infinite}\index{Counting!Countably Infinite} if there exists a bijection \(f:\:S\rightarrow\mathbb{N}\). 
\end{definition}

Intuitively, there is some intuitive way to write \(A\) in an order such that an \(n\)th element can be determined.
This is in the same way how the ninth element of the infinite natural numbers can be determined. It is in this
sense that \(\mathbb{N}\) is described as countable and by extension \(A\). 

\bigskip
\begin{definition}
    A set \(A\) is \emph{countable}\index{Counting!Countable} if it is finite or countably infinite.
\end{definition}

\bigskip
\begin{definition}
    A set \(S\) is \emph{uncountable}\index{Counting!Uncountable} if it is not countable.
\end{definition}

\subsection{Integers are Countably Infinite}

\begin{proposition}
    The integers are countably infinite, i.e.\ \(|\mathbb{Z}|=|\mathbb{N}|\).
\end{proposition}

Consider the following diagram representing the bijection:
\begin{figure}[H]
    \centering
    \textbf{Mapping From \(\mathbb{Z}\) to \(\mathbb{N}\)}

    \resizebox{18em}{7em}{
        \begin{tikzpicture}[line cap=round,line join=round,>=triangle 45,x=1.0cm,y=1.0cm]
            \clip(-2.5,-0.5) rectangle (4.5,2.);
            \draw [<->,>=stealth,line width=1.pt,domain=-2.5:4.5] plot(\x,{(-0.-0.*\x)/6.});
            \draw [<-,>=stealth,line width=1.pt] (0.,0.25) circle (0.25cm);
            \draw [<-,>=stealth,shift={(0.,0.)},line width=1.pt]  plot[domain=0.:3.141592653589793,variable=\t]({1.*1.*cos(\t r)+0.*1.*sin(\t r)},{0.*1.*cos(\t r)+1.*1.*sin(\t r)});
            \draw [<-,>=stealth,shift={(1.5,0.)},line width=1.pt]  plot[domain=0.:3.141592653589793,variable=\t]({1.*0.5*cos(\t r)+0.*0.5*sin(\t r)},{0.*0.5*cos(\t r)+1.*0.5*sin(\t r)});
            \draw [<-,>=stealth,shift={(0.5,-1.3333333333333333)},line width=1.pt]  plot[domain=0.4899573262537283:2.651635327336065,variable=\t]({1.*2.833333333333333*cos(\t r)+0.*2.833333333333333*sin(\t r)},{0.*2.833333333333333*cos(\t r)+1.*2.833333333333333*sin(\t r)});
            \draw [<-,>=stealth,shift={(3.5,0.)},line width=1.pt]  plot[domain=0.:3.141592653589793,variable=\t]({1.*0.5*cos(\t r)+0.*0.5*sin(\t r)},{0.*0.5*cos(\t r)+1.*0.5*sin(\t r)});
            \begin{scriptsize}
            \draw [fill=black] (-2.,0.) circle (1.5pt);
            \draw[color=black] (-2,-0.3) node {$-2$};
            \draw [fill=black] (-1.,0.) circle (1.5pt);
            \draw[color=black] (-1,-0.3) node {$-1$};
            \draw [fill=black] (0.,0.) circle (1.5pt);
            \draw[color=black] (0,-0.3) node {$0$};
            \draw [fill=black] (1.,0.) circle (1.5pt);
            \draw[color=black] (1,-0.3) node {$1$};
            \draw [fill=black] (2.,0.) circle (1.5pt);
            \draw[color=black] (2,-0.3) node {$2$};
            \draw [fill=black] (3.,0.) circle (1.5pt);
            \draw[color=black] (3,-0.3) node {$3$};
            \draw [fill=black] (4.,0.) circle (1.5pt);
            \draw[color=black] (4,-0.3) node {$4$};
            \end{scriptsize}
        \end{tikzpicture}
    }
\end{figure}
Clearly, this is a bijection. The mapping can be represented by the following table:
\begin{figure}[H]
    \centering
    \begin{tabular}{c|c}
        \(\mathbb{Z}\) & \(\mathbb{N}\) \\
        \hline
        \(0\) & \(0\) \\
        \(-1\) & \(1\) \\
        \(1\) & \(2\) \\
        \(-2\) & \(3\) \\
        \(2\) & \(4\)
    \end{tabular}
\end{figure}
The function and its inverse may be explicitly defined:
\begin{align*}
    f:\quad\mathbb{Z}&\longrightarrow\mathbb{N}_0 \\
                    x&\longrightarrow\begin{cases}
                        |2x|-1 & x<0 \\ 2x & x\ge0
                    \end{cases}
\end{align*}
\begin{align*}
    f^{-1}:\quad\mathbb{N}_0&\longrightarrow\mathbb{Z} \\
                           x&\longrightarrow\lfloor(-1)^{x}\frac{x}{2}\rfloor
\end{align*}
Because there is a bijection between \(\mathbb{Z}\) and \(\mathbb{N}\), \(|\mathbb{Z}|=|\mathbb{N}|\).
This means that the integers are countably infinite. 

\subsection{Rationals Numbers are Countably Infinite}

\begin{proposition}
    The rational numbers are countably infinite, i.e.\ \(|\mathbb{Q}|=|\mathbb{N}|\).
\end{proposition}

We can consider rational numbers as a pair of integers. In this way, a fraction can be represented by
a 2-dimensional coordinate of integers. However, two coordinates may represent the same fraction. For
example, take \((p,q)\) and \((2p,2q)\) where \(p,q\in\mathbb{Z}\). This issue is mitigated if points
which their fraction representation would not be in simplified form are ignored. In other words, \((p,q)\)
are not considered when \(p\) and \(q\) are not \emph{coprime}. 

Another consequence is that the third and fourth quadrant of the cartesian coordinate system may be ignored
because all fractions formed by the points in those quadrants can be formed by the points in the first and second
quadrant.

The grid would look like the following:
\begin{figure}[H]
    \centering
    \textbf{Coordinates That Represents Fractions}

    Any fraction \(p/q\) corresponds with point \((p,q)\).

    \resizebox{15em}{8em}{
        \begin{tikzpicture}[line cap=round,line join=round,>=triangle 45,x=1.0cm,y=1.0cm]
            \clip(-3.5,-0.5) rectangle (3.5,3.5);
            \draw [<->,>=stealth,line width=2.pt] (-3.5,0.)-- (3.5,0.);
            \draw [<->,>=stealth,line width=2.pt] (0.,3.5)-- (0.,-0.5);
            \draw (3.190369660517843,0.5361219287454672) node[anchor=north west] {$p$};
            \draw (0.11387172475044952,3.5242604707462206) node[anchor=north west] {$q$};
            \draw (-3,-0.25) node[anchor=center] {$-3$};
            \draw (-2,-0.25) node[anchor=center] {$-2$};
            \draw (-1,-0.25) node[anchor=center] {$-1$};
            \draw (1,-0.25) node[anchor=center] {$1$};
            \draw (2,-0.25) node[anchor=center] {$2$};
            \draw (3,-0.25) node[anchor=center] {$3$};
            \draw [fill=red] (-3.,3.) circle (2.5pt);
            \draw [fill=black] (-2.,3.) circle (2.5pt);
            \draw [fill=black] (-1.,3.) circle (2.5pt);
            \draw [fill=red] (0.,3.) circle (2.5pt);
            \draw (-0.25,3) node[anchor=center] {$3$};
            \draw [fill=black] (1.,3.) circle (2.5pt);
            \draw [fill=black] (2.,3.) circle (2.5pt);
            \draw [fill=red] (3.,3.) circle (2.5pt);
            \draw [fill=black] (-3.,2.) circle (2.5pt);
            \draw [fill=red] (-2.,2.) circle (2.5pt);
            \draw [fill=black] (-1.,2.) circle (2.5pt);
            \draw [fill=red] (0.,2.) circle (2.5pt);
            \draw (-0.25,2) node[anchor=center] {$2$};
            \draw [fill=black] (1.,2.) circle (2.5pt);
            \draw [fill=red] (2.,2.) circle (2.5pt);
            \draw [fill=black] (3.,2.) circle (2.5pt);
            \draw [fill=black] (-3.,1.) circle (2.5pt);
            \draw [fill=black] (-2.,1.) circle (2.5pt);
            \draw [fill=black] (-1.,1.) circle (2.5pt);
            \draw [fill=black] (0.,1.) circle (2.5pt);
            \draw (-0.25,1) node[anchor=center] {$1$};
            \draw [fill=black] (1.,1.) circle (2.5pt);
            \draw [fill=black] (2.,1.) circle (2.5pt);
            \draw [fill=black] (3.,1.) circle (2.5pt);
        \end{tikzpicture}  
    }

    The dots in red are omitted points because the coordinates are not coprime.
\end{figure}

How consider the following order of these points. Each arrow corresponds with a natural number.
This is what makes the rational number countably infinite, and this correspondence defines a
bijection between the natural numbers and the rational numbers.

\begin{figure}[H]
    \centering
    \textbf{An Order of Rationals}

    \resizebox{14em}{8em}{
        \begin{tikzpicture}[line cap=round,line join=round,>=triangle 45,x=1.0cm,y=1.0cm]
            \clip(-2.5,-0.5) rectangle (3.5,3.5);
            \draw [<->,>=stealth,line width=2.pt] (-2.5,0.)-- (3.5,0.);
            \draw [<->,>=stealth,line width=2.pt] (0.,3.5)-- (0.,-0.5);
            \draw (3.190369660517843,0.5361219287454672) node[anchor=north west] {$p$};
            \draw (0.11387172475044952,3.5242604707462206) node[anchor=north west] {$q$};
            \draw (-2,-0.25) node[anchor=center] {$-2$};
            \draw (-1,-0.25) node[anchor=center] {$-1$};
            \draw (1,-0.25) node[anchor=center] {$1$};
            \draw (2,-0.25) node[anchor=center] {$2$};
            \draw (3,-0.25) node[anchor=center] {$3$};
            \draw [->,>=stealth,line width=2.pt] (0.,1.)-- (1.,1.);
            \draw [->,>=stealth,line width=2.pt] (1.,1.)-- (1.,2.);
            \draw [line width=2.pt] (1.,2.)-- (0.,2.);
            \draw [->,>=stealth,line width=2.pt] (0.,2.)-- (-1.,2.);
            \draw [->,>=stealth,line width=2.pt] (-1.,2.)-- (-1.,1.);
            \draw [->,>=stealth,line width=2.pt] (-1.,1.)-- (-2.,1.);
            \draw [line width=2.pt] (-2.,1.)-- (-2.,2.);
            \draw [->,>=stealth,line width=2.pt] (-2.,2.)-- (-2.,3.);
            \draw [->,>=stealth,line width=2.pt] (-2.,3.)-- (-1.,3.);
            \draw [line width=2.pt] (-1.,3.)-- (0.,3.);
            \draw [->,>=stealth,line width=2.pt] (0.,3.)-- (1.,3.);
            \draw [->,>=stealth,line width=2.pt] (1.,3.)-- (2.,3.);
            \draw [line width=2.pt] (2.,3.)-- (2.,2.);
            \draw [->,>=stealth,line width=2.pt] (2.,2.)-- (2.,1.);
            \draw [->,>=stealth,line width=2.pt] (2.,1.)-- (3.,1.);
            \draw [->,>=stealth,line width=2.pt] (3.,1.)-- (3.,2.);
            \draw [line width=2.pt] (3.,2.)-- (3.,3.);
            \draw [fill=black] (-2.,3.) circle (2.5pt);
            \draw [fill=black] (-1.,3.) circle (2.5pt);
            \draw [fill=red] (0.,3.) circle (2.5pt);
            \draw (-0.25,2.8) node[anchor=center] {$3$};
            \draw [fill=black] (1.,3.) circle (2.5pt);
            \draw [fill=black] (2.,3.) circle (2.5pt);
            \draw [fill=red] (3.,3.) circle (2.5pt);
            \draw [fill=red] (-2.,2.) circle (2.5pt);
            \draw [fill=black] (-1.,2.) circle (2.5pt);
            \draw [fill=red] (0.,2.) circle (2.5pt);
            \draw (-0.25,1.8) node[anchor=center] {$2$};
            \draw [fill=black] (1.,2.) circle (2.5pt);
            \draw [fill=red] (2.,2.) circle (2.5pt);
            \draw [fill=black] (3.,2.) circle (2.5pt);
            \draw [fill=black] (-2.,1.) circle (2.5pt);
            \draw [fill=black] (-1.,1.) circle (2.5pt);
            \draw [fill=black] (0.,1.) circle (2.5pt);
            \draw (-0.25,0.8) node[anchor=center] {$1$};
            \draw [fill=black] (1.,1.) circle (2.5pt);
            \draw [fill=black] (2.,1.) circle (2.5pt);
            \draw [fill=black] (3.,1.) circle (2.5pt);
        \end{tikzpicture}
    }
\end{figure}

For example, in this definition, the first rational number is \(1/0\) or \(0\); the fifth rational number
is \(-1\) and the seventh rational number is \(-\frac{2}{3}\).

In conclusion, \(\mathbb{Q}\) is countably infinite and \(|\mathbb{Q}|=|\mathbb{N}|\).

\bigskip
\textbf{Final Notes:}
\begin{itemize}
    \item There are no points on the \(p\) axis because at those points \(q=0\). A fraction \(p/q\) cannot have
    \(0\) as a denominator.
    \item Two numbers are \emph{coprime} if their greatest common factor is \(1\). Fractions are in simplest
    forms when the numerator and denominator are coprime. For this section, I will denote the greatest common
    factor of two integers \(a,b\) as \((a,b)\) as is standard. This should not be confused with coordinates which 
    will be explicitly stated. Thus, given any coordinate \((p,q)\), if \((p,q)\ne1\), then it is a unsimplified fraction
    as it can be reduced by \((p,q)\).
    \item Coordinate \((1,0)\) is included as it is equal to \(0\). However, coordinate \((2,0)\) correspond with
    the rational \(0\). However, \((2,0)=2\) as the greatest common factor of both is \(2\) since \(2\mid 0\).
    Therefore, for points \((p,0)\) for all integers \(p\) not equal to \(0\), the corresponding fraction can be simplified.
\end{itemize}

\subsection{Examples of Uncountable Sets}

\begin{proposition}
    The real numbers \(\mathbb{R}\) is uncountable.
\end{proposition}

\bigskip
\begin{proposition}
    The power set of the natural numbers \(\mathcal{P}(\mathbb{N})\) is uncountable.
\end{proposition}