\lecturechapter{Written by Jeffrey J.}

\section{Introduction to Boolean Algebra}
\index{Boolean Algebra}

Boolean algebra is a branch of algebra that deals with true and false values. It is used to mathematically represent logic.
True is represented by \(T\), and false is represented by \(F\).

There are three fundamental operations in boolean algebra:
\begin{itemize}
    \item \emph{Conjunction}. This is ``logical and'' and it is denoted with \(\land\).
    \item \emph{Disjunction}. This is ``logical or'' and it is denoted with \(\lor\).
    \item \emph{Negation}. This is ``logical not'' and it is denoted with \(\lnot\).
\end{itemize}

The truth tables of these basic operation are shown below:

\begin{figure}[H]
    \centering
    Truth Table for Conjunction

    \begin{tabular}{ccc}
        \(x\) & \(y\) & \(x\land y\) \\
        \hline
        \(T\) & \(T\) & \(T\) \\
        \(T\) & \(F\) & \(F\) \\
        \(F\) & \(T\) & \(F\) \\
        \(F\) & \(F\) & \(F\)
    \end{tabular}
\end{figure}

\begin{figure}[H]
    \centering
    Truth Table for Disjunction

    \begin{tabular}{ccc}
        \(x\) & \(y\) & \(x\lor y\) \\
        \hline
        \(T\) & \(T\) & \(T\) \\
        \(T\) & \(F\) & \(T\) \\
        \(F\) & \(T\) & \(T\) \\
        \(F\) & \(F\) & \(F\)
    \end{tabular}
\end{figure}

\begin{figure}[H]
    \centering
    Truth Table for Negation

    \begin{tabular}{cc}
        \(x\) & \(\lnot x\) \\
        \hline
        \(T\) & \(F\) \\
        \(F\) & \(T\) 
    \end{tabular}
\end{figure}

\section{Introduction to Set Theory, Continued.}
\index{Set Theory}

\subsection{Set-Builder Notation}
\index{Set Theory!Set-Builder Notation}

\emph{Set-Builder Notation} is a notation used to describe the elements of a set. In its most abstract form, it can be represented as \(\set{x\mid\lambda(x)}\).
\(x\) is the element that would make up the set. The \(\mid\) symbol represents ``such that'' and \(\lambda(x)\) is a \emph{predicate} (a function that returns true or false).
\(\lambda\) defines the properties that \(x\) needs to be included in the set. Additionally, you may specify the domain of the elements of the set. For example, 
\(\set{x\in\mathbb{Z}\mid\lambda(x)}\) meaning the set contains all elements of the set of integers that satisfy \(\lambda\).\footnote{Some set-builder notation
uses \(:\) instead of \(\mid\) for ``such that.''}

Examples: 
\begin{itemize}
    \item \(\mathbb{C}=\setbuild{a+bi}{a,b\in\mathbb{R},i=\sqrt{-1}}\). 
    \begin{itemize}
        \item The complex numbers are the set whose elements are in the form \(a+bi\) such that \(a,b\) are real numbers and \(i\) is the square root of \(-1\).
    \end{itemize}
    \item \(K=\setbuild{x}{\sqrt{x}\in\mathbb{Z}}\). 
    \begin{itemize}
        \item This is the set of integer squares. This reads \(K\) is the set that has elements \(x\) such that the square root of \(x\) is an integer. 
    \end{itemize}
    \item \(\mathbb{E}=\set{x\in\mathbb{Z}:2\mid x}\). Here I use \(:\) for ``such that'' to differentiate from the ``divides'' operator. 
    \begin{itemize}
        \item This is the set of evens. This reads that \(\mathbb{E}\) is the set of integers where \(2\) evenly divides the integer.
    \end{itemize}
    \item \(\mathcal{P}(S)=\setbuild{X}{X\subseteq S}\). 
    \begin{itemize}
        \item The power set of \(S\) is all the sets \(X\) such that \(X\) is a subset of \(S\).
    \end{itemize}
    \item \(S\times R=\setbuild{(s,r)}{s\in S,r\in R}\). This set is called the cartesian product of set \(S\) and \(R\). For example, \(\mathbb{R}^2=\mathbb{R}\times\mathbb{R}\) 
    is set of pairs of real numbers. 
\end{itemize}

More complex sets can be constructed in set builder notation in conjunction with the logical operators. For example, \(\setbuild{x\in\mathbb{Z}}{0<x<12\land x\omod 2=1}\) is the 
set of odd integers between \(0\) and \(12\).

\subsection{More Set Operations}
\index{Set Theory!Set Operations}

\bigskip
\begin{definition}
    The \emph{union} of two sets is the set containing all the elements of both sets. It is denoted with \(\cup\), and it is represented as \(A\cup B=\setbuild{x}{x\in A\lor x\in B}\).
\end{definition}

\bigskip
\begin{definition}
    The \emph{intersection} of two sets is the set whose elements are in both sets. It is denoted with \(\cap\), and it is represented as \(A\cap B=\setbuild{a\in A}{a\in B}\) or
    \(\setbuild{x}{x\in A\land x\in B}\).
\end{definition}

\bigskip
\begin{definition}
    The \emph{set-difference} of two sets, denoted \(A-B\) or \(A\backslash\) with sets \(A,B\), is all the elements of \(A\) that are not in \(B\). In set-builder notation,
    \(A-B=\setbuild{a\in A}{a\not\in B}\) or \(A-B=\setbuild{x}{x\in A\land x\not\in B}\).
\end{definition}

There was two different representation of intersection and set difference in set-builder notation provided. Although they both describe the same action, they have different semantic meaning.
Take intersection, \(A\cap B=\setbuild{a\in A}{b\in B}\) means is the set of elements of \(A\) such that they are also in \(B\). However, \(\setbuild{x}{x\in A\land x\in B}\) means the set 
of values such that the value are both contained in \(A\) and \(B\). What the ``values'' are is unstated. Typically, these values are assumed, or the domain was previously established. The
``values'' are implied.

Examples:
\begin{itemize}
    \item \(\set{a,b}\cup\set{b,c}=\set{a,b,c}\)
    \item \(\set{a,b}\cap\set{b,c}=\set{b}\)
    \item \(\set{a,b}-\set{b,c}=\set{a}\)
    \item \(S\cup\emptyset=S\)
    \item \(S\cap\emptyset=\emptyset\)
    \item \(S-\emptyset=S\)
    \item \(\emptyset-S=\emptyset\)
    \item \(\mathbb{R}-\mathbb{Q}\) is the set of irrational numbers.
\end{itemize}

\bigskip
\begin{definition}
    The \emph{complement} of a set \(S\) is the set of values which are not in \(S\). The domain of set of ``values'' is implied or was previously established. It is typically denoted
    by \(\overline{S}\). 
\end{definition}

The complement of a set is a specific case of set difference. Namely, if \(\overline{S}\) is the complement to set \(R\), then \(\overline{S}=R-S\). Importantly, \(S\subseteq R\). 
This last requirement for complement is not required for regular set difference (e.g.\ \(\set{a,b}-\set{b,c}=\set{a}\) and \(\set{b,c}\not\subseteq\set{a,b}\)).

Examples:
\begin{itemize}
    \item The complement of \(\set{2,3,4}\) to \(\set{1,2,3,4,5}\) is \(\overline{\set{2,3,4}}=\set{1,5}\).
    \item Given that \(\mathbb{E}\) is the set of even integers, \(\overline{\mathbb{E}}=\mathbb{O}\) where \(\mathbb{O}\) is the set of odd integers. This assumes that it is the complement
    to the set of integers \(\mathbb{Z}\).
\end{itemize}

\bigskip
\begin{definition}
    Two sets are \emph{disjoint} if they do not share any elements, i.e.\ the intersection of the sets is the empty set.
\end{definition}

A simple example is that \(\mathbb{E}\) is disjoint to \(\mathbb{O}\) where they denote even and odd integers respectively.

\bigskip
\begin{definition}
    The \emph{cardinality} or \emph{order} of a set is the number of unique elements in the set.
\end{definition}

Examples:
\begin{itemize}
    \item \(|\set{1,2,3}|=3\)
    \item \(|\set{\set{1,2,3}}|=1\)
    \item \(|\emptyset|=0\)
    \item \(|\mathbb{Z}|=\infty\)
    \item \(|\symgroup{n}|=n!\). \(\text{Sym}_n\) is the set of permutations on set \(\setbuild{x\in\mathbb{Z}}{0<x\le n}=\set{1,2,\dots,n}\), i.e.\ the 
    different ways which the set can be rearranged.  
    \item The order of the set of symmetries on a regular \(n\)-sided polygon is \(2n\), denoted as \(|D_{2n}|=2n\).
\end{itemize}

\bigskip
\begin{definition}
    The \emph{power set} of set \(S\) denoted with \(\mathcal{P}(S)\) is the set of all subsets of \(S\). In set builder notation, it is \(\mathcal{P}(S)=\setbuild{X}{X\subseteq S}\).
\end{definition}

Examples:
\begin{itemize}
    \item \(\mathcal{P}(\set{1,2})=\set{\emptyset,\set{1},\set{2},\set{1,2}}\)
    \item \(\mathcal{P}(\emptyset)=\set{\emptyset}\)
\end{itemize}