\lecturechapter{Written by Jeffrey J.}

\section{Counting and Infinity, Continued}
\index{Counting}

\subsection{Set Operations on Countability}
\index{Set Theory!Set Operations}

\begin{proposition}
    If set \(S_1\) and \(S_2\) are countably infinite, then \(S_1\cup S_2\) is countably infinite.
\end{proposition}
\begin{proof}
    Consider the following order of elements of \(S_1\) and \(S_2\).
    \begin{figure}[H]
        \centering
        \resizebox{15em}{5em}{
            \begin{tikzpicture}[line cap=round,line join=round,>=triangle 45,x=1.0cm,y=1.0cm]
                \clip(-1.,-0.5) rectangle (5.,1.5);
                \draw (-0.5,0) node[anchor=center] {$S_1$};
                \draw (-0.5,1) node[anchor=center] {$S_2$};
                \draw [->,>=stealth,line width=2.pt] (0.,0.)-- (0.,1.);
                \draw [->,>=stealth,line width=2.pt] (0.,1.)-- (1.,0.);
                \draw [->,>=stealth,line width=2.pt] (1.,0.)-- (1.,1.);
                \draw [->,>=stealth,line width=2.pt] (1.,1.)-- (2.,0.);
                \draw [->,>=stealth,line width=2.pt] (2.,0.)-- (2.,1.);
                \draw [->,>=stealth,line width=2.pt] (2.,1.)-- (3.,0.);
                \draw [->,>=stealth,line width=2.pt] (3.,0.)-- (3.,1.);
                \draw [->,>=stealth,line width=2.pt] (3.,1.)-- (4.,0.);
                \draw [->,>=stealth,line width=2.pt] (4.,0.)-- (4.,1.);
                \draw [->,>=stealth,line width=2.pt] (4.,1.)-- (4.5,0.5);
                \begin{scriptsize}
                \draw [fill=black] (0.,0.) circle (2.5pt);
                \draw [fill=black] (0.,1.) circle (2.5pt);
                \draw [fill=black] (1.,1.) circle (2.5pt);
                \draw [fill=black] (1.,0.) circle (2.5pt);
                \draw [fill=black] (2.,0.) circle (2.5pt);
                \draw [fill=black] (2.,1.) circle (2.5pt);
                \draw [fill=black] (3.,1.) circle (2.5pt);
                \draw [fill=black] (3.,0.) circle (2.5pt);
                \draw [fill=black] (4.,1.) circle (2.5pt);
                \draw [fill=black] (4.,0.) circle (2.5pt);
                \end{scriptsize}
            \end{tikzpicture}
        }
    \end{figure}
    Because there is a method of ordering the elements of \(S_1\) and \(S_2\), there is a bijection from \(S_1\cup S_2\) and the natural numbers \(\mathbb{N}\).
    This bijection means that \(S_1\cup S_2\) is countably infinite.
\end{proof}

\bigskip
\begin{proposition}
    If set \(S_1\) and \(S_2\) are countably infinite, then \(S_1-S_2\) is countable.
\end{proposition}
\begin{proof}
    Consider the case when \(S_1\ne S_2\). It is possible that \(S_1-S_2\) is finite or it is infinitely countable. The infinite countability can be shown by the
    below diagram.
    \begin{figure}[H]
        \centering
        \resizebox{15em}{5em}{
            \begin{tikzpicture}[line cap=round,line join=round,>=triangle 45,x=1.0cm,y=1.0cm]
                \clip(-1.75,-0.5) rectangle (4.5,1.5);
                \draw (-1,0) node[anchor=center] {$S_1-S_2$};
                \draw (-0.5,1) node[anchor=center] {$S_2$};
                \draw [line width=2.pt] (0.,0.)-- (1.,0.);
                \draw [->,>=stealth,line width=2.pt] (1.,0.)-- (2.,0.);
                \draw [line width=2.pt] (2.,0.)-- (3.,0.);
                \draw [->,>=stealth,line width=2.pt] (3.,0.)-- (4.,0.);
                \begin{scriptsize}
                \draw [fill=black] (0.,0.) circle (2.5pt);
                \draw [fill=black] (1.,1.) circle (2.5pt);
                \draw [fill=red] (1.,0.) circle (2.5pt);
                \draw [fill=black] (2.,0.) circle (2.5pt);
                \draw [fill=black] (3.,1.) circle (2.5pt);
                \draw [fill=red] (3.,0.) circle (2.5pt);
                \draw [fill=black] (4.,0.) circle (2.5pt);
                \end{scriptsize}
            \end{tikzpicture}
        }
    \end{figure}
    Additionally, the case that \(S_1=S_2\) shows that \(S_1-S_2=\emptyset\) meaning \(S_1-S_2\) is finite. Therefore, \(S_1-S_2\) is countable, but it may be either
    finite or infinite.
\end{proof}

\newpage
\begin{proposition}
    If set \(S_1\) and \(S_2\) are countably infinite, then \(S_1\times S_2\) is countably infinite.
\end{proposition}
\begin{proof}
    Consider the countably infinite sets \(S,T\). Since they are countable, \(S,T\) may be enumerated and \(S\times T\) has elements \((S_i,T_j)\) for some \(i,j\in\mathbb{N}\).
    A new order may be defined as follows:
    \begin{figure}[H]
        \centering
        \resizebox{15em}{13em}{
            \begin{tikzpicture}[line cap=round,line join=round,>=triangle 45,x=1.0cm,y=1.0cm]
                \clip(-1.,-1.) rectangle (5.,4.5);
                \draw [->,>=stealth,line width=2.pt] (0.,0.)-- (0.,4.);
                \draw [->,>=stealth,line width=2.pt] (0.,0.)-- (4.,0.);
                \draw [->,>=stealth,line width=2.pt] (1.,1.)-- (2.,1.);
                \draw [->,>=stealth,line width=2.pt] (2.,1.)-- (1.,2.);
                \draw [->,>=stealth,line width=2.pt] (1.,2.)-- (1.,3.);
                \draw [->,>=stealth,line width=2.pt] (1.,3.)-- (2.,2.);
                \draw [->,>=stealth,line width=2.pt] (2.,2.)-- (3.,1.);
                \draw [->,>=stealth,line width=2.pt] (4.,1.)-- (3.,2.);
                \draw [->,>=stealth,line width=2.pt] (3.,2.)-- (2.,3.);
                \draw [->,>=stealth,line width=2.pt] (2.,3.)-- (1.,4.);
                \draw [->,>=stealth,line width=2.pt] (2.,4.)-- (3.,3.);
                \draw [->,>=stealth,line width=2.pt] (3.,3.)-- (4.,2.);
                \draw [->,>=stealth,line width=2.pt] (3.,1.)-- (4.,1.);
                \draw (3.75,0.4) node[anchor=center] {$S$};
                \draw (0.4,3.75) node[anchor=center] {$T$};
                \draw (-0.3,1) node[anchor=center] {$T_1$};
                \draw (-0.3,2) node[anchor=center] {$T_2$};
                \draw (-0.3,3) node[anchor=center] {$T_3$};
                \draw (1,-0.3) node[anchor=center] {$S_1$};
                \draw (2,-0.3) node[anchor=center] {$S_2$};
                \draw (3,-0.3) node[anchor=center] {$S_3$};
                \begin{scriptsize}
                \draw [fill=black] (1.,1.) circle (2.5pt);
                \draw [fill=black] (2.,1.) circle (2.5pt);
                \draw [fill=black] (3.,1.) circle (2.5pt);
                \draw [fill=black] (1.,2.) circle (2.5pt);
                \draw [fill=black] (2.,2.) circle (2.5pt);
                \draw [fill=black] (3.,2.) circle (2.5pt);
                \draw [fill=black] (1.,3.) circle (2.5pt);
                \draw [fill=black] (2.,3.) circle (2.5pt);
                \draw [fill=black] (3.,3.) circle (2.5pt);
                \end{scriptsize}
            \end{tikzpicture}
        }
    \end{figure}
    Because there is a defined order, there is a bijection between \(S\times T\) and the natural numbers \(\mathbb{N}\). Therefore, \(S\times T\) is countably infinite.
\end{proof}

\bigskip
\begin{proposition}
    If set \(S\) is countably infinite and \(T\) is finite, then \(S\times T\) is countable.
\end{proposition}
\begin{proof}
    Consider the case when \(T\ne\emptyset\). By a similar process as the above proposition, \(S\times T\) can be ordered by the \(T\) element in the pair. As a result,
    when \(T\ne\emptyset\), \(S\times T\) is countably infinite. However, when \(T=\emptyset\), \(S\times T=\emptyset\) which is finite and countable. Therefore,
    \(S\times T\) is countable.
\end{proof}

\bigskip
\begin{proposition}
    Let \(S_1,S_2,\dots\) be countably infinite sets. Then, \(S_1\cup S_2\cup\dots\) is countably infinite. In other words, the union of countably infinite number of
    countably infinite sets is countably infinite.
\end{proposition}
\begin{proof}
    I will provide two proofs. The first proof will create an order from the sets as follows:
    \begin{figure}[H]
        \centering
        \resizebox{16em}{12em}{
            \begin{tikzpicture}[line cap=round,line join=round,>=triangle 45,x=1.0cm,y=1.0cm]
                \clip(-0.5,0.) rectangle (5.5,4.5);
                \draw (0.25,4) node[anchor=center] {$S_1$};
                \draw (0.25,3) node[anchor=center] {$S_2$};
                \draw (0.25,2) node[anchor=center] {$S_3$};
                \draw (0.25,1) node[anchor=center] {$S_4$};
                \draw (0.25,0.5) node[anchor=center] {$\vdots$};
                \draw (1,0.5) node[anchor=center] {$\vdots$};
                \draw (2,0.5) node[anchor=center] {$\vdots$};
                \draw (3,0.5) node[anchor=center] {$\vdots$};
                \draw (4,0.5) node[anchor=center] {$\vdots$};
                \draw (4.75,0.5) node[anchor=center] {$\ddots$};
                \draw (4.75,4) node[anchor=center] {$\dots$};
                \draw (4.75,3) node[anchor=center] {$\dots$};
                \draw (4.75,2) node[anchor=center] {$\dots$};
                \draw (4.75,1) node[anchor=center] {$\dots$};
                \draw [->,>=stealth,line width=2.pt] (1.,4.)-- (2.,4.);
                \draw [->,>=stealth,line width=2.pt] (2.,4.)-- (1.,3.);
                \draw [->,>=stealth,line width=2.pt] (1.,3.)-- (1.,2.);
                \draw [->,>=stealth,line width=2.pt] (1.,2.)-- (2.,3.);
                \draw [->,>=stealth,line width=2.pt] (2.,3.)-- (3.,4.);
                \draw [->,>=stealth,line width=2.pt] (3.,4.)-- (4.,4.);
                \draw [->,>=stealth,line width=2.pt] (4.,4.)-- (3.,3.);
                \draw [->,>=stealth,line width=2.pt] (3.,3.)-- (2.,2.);
                \draw [->,>=stealth,line width=2.pt] (2.,2.)-- (1.,1.);
                \draw [->,>=stealth,line width=2.pt] (2.,1.)-- (3.,2.);
                \draw [->,>=stealth,line width=2.pt] (3.,2.)-- (4.,3.);
                \draw [->,>=stealth,line width=2.pt] (4.,2.)-- (3.,1.);
                \begin{scriptsize}
                \draw [fill=black] (1.,1.) circle (2.5pt);
                \draw [fill=black] (2.,1.) circle (2.5pt);
                \draw [fill=black] (3.,1.) circle (2.5pt);
                \draw [fill=black] (4.,1.) circle (2.5pt);
                \draw [fill=black] (1.,2.) circle (2.5pt);
                \draw [fill=black] (2.,2.) circle (2.5pt);
                \draw [fill=black] (3.,2.) circle (2.5pt);
                \draw [fill=black] (4.,2.) circle (2.5pt);
                \draw [fill=black] (1.,3.) circle (2.5pt);
                \draw [fill=black] (2.,3.) circle (2.5pt);
                \draw [fill=black] (3.,3.) circle (2.5pt);
                \draw [fill=black] (4.,3.) circle (2.5pt);
                \draw [fill=black] (1.,4.) circle (2.5pt);
                \draw [fill=black] (2.,4.) circle (2.5pt);
                \draw [fill=black] (3.,4.) circle (2.5pt);
                \draw [fill=black] (4.,4.) circle (2.5pt);
                \end{scriptsize}
            \end{tikzpicture}
        }
        
    \end{figure}
    Because there is an order, there is a bijection from \(S_1\cup S_2\cup\dots\) to \(\mathbb{N}\) meaning \(S_1\cup S_2\cup\dots\) is countably infinite.

    The second proof will be by induction. It was respectively given and proven that \(S_1\) and \(S_1\cup S_2\) is countably infinite. Assume for \(0<n<N\) that
    \(\bigcup_{i=1}^{n}S_i\) is countably infinite. Therefore, by assumption, \(\bigcup_{i=1}^{N-1}S_i\) is countably infinite. Consider 
    \(S_1\cup S_2\cup\dots\cup S_{N-1}\cup S_{N}\). This is equivalent to \((\bigcup_{i=1}^{N-1}S_i)\cup S_N\). Since both \(\bigcup_{i=1}^{N-1}S_i\) and \(S_N\) are
    both countably infinite, then \((\bigcup_{i=1}^{N-1}S_i)\cup S_N\) is as well. This proves the inductive assumption that the countably infinite number of countably
    infinite sets has an union that is countably infinite.
\end{proof}

\bigskip
\textbf{A note:} A path or order between elements must be able to reach any element in a finite amount of time. This finite time may large, but this means that the path will
theoretically reach every value in the set. In other words, the path can reach all elements eventually. As a result, there may be multiple paths that performs this action,
but not all paths do this. For example, for the above proposition, a path which counts along one of the countably infinite sets will not satisfy the requirements as other
elements in other countably infinite sets will never be reached.

\subsection{The Power Set of Natural Numbers is Uncountable}

\begin{proposition}
    \(\mathcal{P}(\mathbb{N})\) is uncountable.
\end{proposition}
\begin{proof}
    This is a proof by contradiction. Assume that \(\mathcal{P}(\mathbb{N})\) is countably infinite, meaning there is some bijection \(f\) exists between 
    \(\mathcal{P}(\mathbb{N})\leftrightarrow\mathbb{N}\) and the subsets of \(\mathbb{N}\) can be enumerated. Define \(\mathcal{P}(\mathbb{N})=\set{S_1,S_2,\dots}\) 
    where \(\forall i\ S_i\subseteq\mathbb{N}\) and has all the subsets.

    If we represent each substring as a bit-string similarly to how the order of the power set was proven, then it may be represented as a table.
    \begin{figure}[H]
        \centering
        \begin{tabular}{c|cccccc}
                    & \(1\) & \(2\) & \(3\) & \(4\) & \(5\) & \(\dots\) \\
            \hline
            \(S_1\) & \textcolor{blue}{\(0\)} & \(1\) & \(0\) & \(1\) & \(1\) & \(\dots\) \\
            \(S_2\) & \(0\) & \textcolor{blue}{\(0\)} & \(0\) & \(0\) & \(0\) & \(\dots\) \\
            \(S_3\) & \(0\) & \(1\) & \textcolor{blue}{\(1\)} & \(1\) & \(1\) & \(\dots\) \\
            \(S_4\) & \(1\) & \(1\) & \(0\) & \textcolor{blue}{\(1\)} & \(0\) & \(\dots\) \\
            \(S_5\) & \(1\) & \(0\) & \(0\) & \(0\) & \textcolor{blue}{\(0\)} & \(\dots\) \\
            \(\vdots\) & \(\vdots\) & \(\vdots\) & \(\vdots\) & \(\vdots\) & \(\vdots\) & \(\ddots\)
        \end{tabular}
    \end{figure}
    Consider the negation of the bit-string formed by the diagonal, \(T=11001\dots\). It is noted that \(\forall i\ T\ne S_i\) as the \(i\)th bit would be opposites between
    \(S_i\) and \(T\). Mathematically, \(T\) is defined as follows, \(T=\setbuild{i}{i\notin S_i}\). \(\forall i\ T\ne S_i\) as \(T\) contains elements that are not in any
    \(S_i\). This contradicts the that \(\mathcal{P}(\mathbb{N})\) contains all of the subsets of \(\mathbb{N}\) as \(T\) is one such subset not included in \(\mathcal{P}(\mathbb{N})\).
    In conclusion, there is no bijection between the power set on natural numbers to the natural numbers, and \(\mathcal{P}(\mathbb{N})\) is uncountable. This is called the 
    diagonalization argument.
\end{proof}