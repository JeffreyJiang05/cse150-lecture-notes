\lecturechapter{Written by Jeffrey J.}

\section{Introduction to Boolean Algebra, Continued.}
\index{Boolean Algebra}

\subsection{More Notes on Quantifiers}

The order of quantifiers matter! For example, take Goldbach's Conjecture.

\begin{proposition}
    Every even integer greater than \(2\) is the sum of two prime numbers.\footnote{Here prime numbers are defined to be the 
    positive prime elements in the integers. Colloquially, prime numbers are positive, but the mathematical definition of prime
    includes the negatives, i.e.\ \(p\) is prime if and only if \(p\mid ab\), then either \(p\mid a\) or \(p\mid b\).} 
\end{proposition}
This can be written in propositional logic as follows: Let \(\mathbb{E}\) be the set of even numbers greater than \(2\). Let \(P\) be the set of
prime numbers. Goldbach's conjecture states \(\forall\:x\in\mathbb{E}\:\exists\:a,b\in P\:a+b=x\). However, switching the quantifiers would make
the statement \(\exists\:a,b\in P\:\forall\:x\in\mathbb{E}\:a+b=x\). This means that there exists two primes \(a,b\) where their sum is equal to
all the even integers greater than \(2\). This is clearly incorrect and has a different meaning from Goldbach's conjecture. 

More specifically, the order of quantifiers matter between different quantifiers. If two quantifiers next to each other are the same, then they can 
be readily switched. For example, consider \(\forall\:a,b\) which is the same as \(\forall\:a\:\forall\:b\). This is equivalent to 
\(\forall\:b\:\forall\:a\).

\bigskip
This is how negation affects existential quantifiers:
\begin{itemize}
    \item \(\lnot(\forall\:x\:P(x))\Rightarrow\:\exists\:x\:\lnot P(x)\)
    \item \(\lnot(\exists\:x\:P(x))\Rightarrow\:\forall\:x\:\lnot P(x)\)
\end{itemize}
An example of simplifying a proposition containing quantifiers is shown below:
\begin{align*}
    \lnot(\forall\:x\:\exists\:y\:\forall\:z\:P(x,y,z))&=\exists\:x\:\lnot(\exists\:y\:\forall\:z\:P(x,y,z))\\
                                                       &=\exists\:x\:\forall\:y\:\lnot(\forall\:z\:P(x,y,z))\\
                                                       &=\exists\:x\:\forall\:y\:\exists\:z\:\lnot P(x,y,z)
\end{align*}

\bigskip
\textbf{A note:} The \(\forall\) can be thought as a sequence of \(\land\) which require that all \(x\) to satisfy the predicate. In notation,
\[(\forall\:x\in S\:P(x))\equiv(\bigwedge_{x\in S}P(x))\]
Similarly, the \(\exists\) can be thought as a sequence or \(\lor\) which require that at least a single \(x\) to satisfy the predicate. In notation,
\[(\exists\:x\in S\:P(x))\equiv(\bigvee_{x\in S}P(x))\]
Here the large conjunction and disjunction represents performing conjunction and disjunction in a sequence. This is akin to how \(\sum\) is used for
summation and \(\prod\) is used for multiplying in a sequence.

\bigskip
\textbf{Another note:} Given \(\forall\:x\in S\:P(x)\), it means \(\forall\:x\:(x\in S)\rightarrow P(x)\). However, given \(\exists\:x\in S\:P(x)\),
it means \(\exists\:x\:(x\in S)\land P(x)\).

\bigskip
\subsection{Limits on Symbolic Logic}
There are sometimes limits on symbolic logic. This usually occurs when the it is self-referential. Take the following example:

Let proposition \(P\) be ``This sentence contains a double negative.'' This is a false statement. However, consider \(\lnot\lnot P\) to be
``It is not the case that the sentence does not contains a double negative.'' This is a true statement. However, \(\lnot\lnot P\equiv P\)
which is false since one is true and the other is false. 

