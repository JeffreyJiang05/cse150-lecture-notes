\lecturechapter{Written by Jeffrey J.}

\section{Introduction to Set Theory}
\index{Set Theory}

\subsection{Basic Definitions and Examples}

\begin{definition}
    A \emph{set}\index{Set Theory!Set} is a grouping of mathematical objects. 
\end{definition}

A set is an unordered collection where the multiplicity,i.e.\ number of times the element occurs, of elements does not matter.

Some important common sets:\index{Set Theory!Common Sets}
\begin{itemize}
    \item The set of natural numbers, \(\mathbb{N}=\set{1,2,3,\dots}\)\footnote{Some include \(0\) as part of the natural numbers.}
    \item The set of integers, \(\mathbb{Z}=\set{\dots,-2,-1,0,1,2,\dots}\)
    \begin{itemize}
        \item Sometimes, positive and negative integers are denoted as \(\mathbb{Z}^+\) and \(\mathbb{Z}^-\) respectively.
        \item Nonnegative integers can be denoted as \(\mathbb{Z}_{\ge0}\).
    \end{itemize}
    \item The set of rational numbers, \(\mathbb{Q}=\setbuild{\frac{a}{b}}{a,b\in\mathbb{Z},b\ne0}\)
    \item The set of irrational numbers, \(\mathbb{R}-\mathbb{Q}\)
    \item The set of complex numbers, \(\mathbb{C}=\setbuild{a+bi}{a,b\in\mathbb{R},i=\sqrt{-1}}\)
\end{itemize}
Some more uncommon sets:
\begin{itemize}
    \item The set of Gaussian integers, \(\mathbb{Z}[i]=\setbuild{a+bi}{a,b\in\mathbb{Z},i=\sqrt{-1}}\)
    \item Complete set of residue classes modulo \(n\), \(\mathbb{Z}/n\mathbb{Z}\).
    \begin{itemize}
        \item The element of \(\mathbb{Z}/n\mathbb{Z}\) is the set of integers \(\mathbb{Z}\) whose remainders are the same
        when divided by \(n\). 
        \item e.g.\ \(\mathbb{Z}/2\mathbb{Z}\) is the set containing the set of even numbers and the set of odd numbers.
        \item This is an example of a set containing sets.
    \end{itemize}
    \newpage
    \item The set of integer polynomials, i.e. polynomials with integer coefficients and variable \(x\), \(\mathbb{Z}[x]\)
    \begin{itemize}
        \item Likewise, the set of real and complex polynomials are \(\mathbb{R}[x]\) and \(\mathbb{C}[x]\) respectively.
        \item These are examples of sets containing polynomials.
    \end{itemize}
    \item The set of invertible \(n\times n\) matrices with real entries, \(\lingroup{n}{\mathbb{R}}\)
    \begin{itemize}
        \item This is an example of a set containing matrices.
    \end{itemize}
    \item The set of permutations on set \(S\), \(\symgroup{S}\)
    \begin{itemize}
        \item A permutation is a function that ``rearranges" the order of the set.
        \item This is an example of a set containing functions.
    \end{itemize}
\end{itemize}
Sets are flexible, and they can contain many more mathematical constructs than listed above.

\bigskip
\begin{definition}
    The \emph{empty set}\index{Set Theory!Empty Set} \(\emptyset\) is the set without any elements. 
\end{definition}


\subsection{Set Operations}
\index{Set Theory!Set Operations}
\begin{definition}
    Two sets are \emph{equal} if they contain the same elements. Likewise, two sets are \emph{not equal} 
    if they do not contain the same elements. Set equality and inequality is denoted by \(=\) and \(\ne\) respectively.   
\end{definition}

\bigskip
\begin{definition}
    If set \(S\) \emph{contains} element \(x\), then \(x\in S\). 
\end{definition}

\bigskip
\begin{definition}
    Set \(A\) is a \emph{subset} of set \(B\) if set \(B\) contains all elements of \(A\), i.e.\ \newline\(a\in B\sforall a\in A\).
    This is denoted as \(A\subseteq B\).
\end{definition}

Examples:
\begin{itemize}
    \item \(\set{1,2}\subseteq\set{1,2,3}\)
    \item \(\set{2,4}\not\subseteq\set{1,2,3}\)
    \item \(2\not\subseteq\set{1,2,3}\) but \(2\in\set{1,2,3}\)
    \item \(\mathbb{N}\subseteq\mathbb{Z}\subseteq\mathbb{Q}\subseteq\mathbb{R}\)
    \item \(\mathbb{R}-\mathbb{Q}\subseteq\mathbb{R}\)
    \item \(\mathbb{Z}[i]\subseteq\mathbb{C}\)
    \item \(\emptyset\subseteq\set{-2,4,7,0}\)
\end{itemize}

\newpage
\bigskip
\begin{proposition}
    The empty set \(\emptyset\) is a subset of any set \(S\).
\end{proposition}
\begin{proof}
    Since the empty set does not contain any elements, by definition, \(S\) contains all of its elements. Therefore, \(\emptyset\subseteq S\).
\end{proof}

\bigskip
\begin{proposition}
    Given sets \(A,B\), \(A=B\) if and only if \(A\subseteq B\) and \(B\subseteq A\).
\end{proposition}
\begin{proof}
    Given \(A=B\). Then, by definition, all elements in \(A\) is contained in element \(B\). Therefore, \(A\subseteq B\).
    Additionally, all elements of \(B\) are contained in \(A\). So, \(B\subseteq A\).

    Now to prove the converse. Given \(A\subseteq B\) and \(B\subseteq A\), assume that \(A\ne B\). This means that there exists an element
    in \(A\) or \(B\) which is not contained in the other set. Consider the case where this element \(x\) is in \(A\) but not \(B\), i.e.\
    \(x\in A\), \(x\not\in B\). By definition, the statement \(A\subseteq B\) implies that all elements of \(A\) are contained in \(B\). This
    is a contradiction. The same contradiction is reached if \(x\in B\) and \(x\not\in A\). Therefore, if \(A\subseteq B\) and \(B\subseteq A\),
    then \(A=B\).
\end{proof}

\bigskip

\begin{definition}
    Set \(A\) is a \emph{proper subset} of set \(B\) if \(A\) is a subset of \(B\) and \(A\) is not equal to \(B\). This is denoted as \(A\subset B\).
\end{definition}

A corollary of proposition \(2.1\) is the empty set \(\emptyset\) is a proper subset of any set \(S\) as long as \(S\ne\emptyset\).

The symbol \(\subseteq\) can be interpreted as “proper subset or equal.” This train of thought
is analogous to how \(\le\) is equivalent to ``less than or equal'' (Note the bottom
of the equals sign below both). Similarly how \(x<y\) implies \(x\le y\), \(A\subset B\)
implies \(A\subseteq B\). By this token, it is clear that for all set \(S\) that \(S\subseteq S\) as S \(S=S\)

\subsection*{Tricky Problems}

\begin{itemize}
    \item \(2\in\set{1,2,3}\), \(\set{2}\not\in\set{1,2,3}\), \(\set{2}\subset\set{1,2,3}\)
    \item \(\emptyset\in\set{\emptyset}\), \(\emptyset\subseteq\set{\emptyset}\), \(\emptyset\subseteq\emptyset\), \(\emptyset=\emptyset\), \(\emptyset\not\in\emptyset\)
\end{itemize}