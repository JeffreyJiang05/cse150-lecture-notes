\lecturechapter{Written by Jeffrey J.}

\section{Set Theory: Relations}
\index{Set Theory!Relations}

\begin{definition}
    A \emph{binary relation} \(R\) on set \(S\) is a subset of \(S\times S=S^2\), i.e.\ \(R\subseteq S\times S\).
\end{definition}

\bigskip
\subsection{Basic Properties of Relations}

\begin{definition}
    A relation \(R\) on set \(S\) is \emph{reflexive} if \(\sforall a\in S\:(a,a)\in R\).
\end{definition}

\bigskip
\begin{definition}
    A relation \(R\) on set \(S\) is \emph{symmetric} if \(\sforall a,b\in S\:(a,b)\in R\Rightarrow(b,a)\in R\).
\end{definition}

\bigskip
\begin{definition}
    A relation \(R\) on set \(S\) is \emph{transitive} if \(\sforall a,b,c\in S\:(a,b)\in R\land(b,c)\in R\Rightarrow(a,c)\in R\).
\end{definition}

\subsubsection{Written Examples}
Let \(S=\) the set of people. Let relation \(R=\) ``is brother of.'' This relation is neither reflexive, symmetric, nor transitive:
\begin{itemize}
    \item \(R\) is not reflexive because it is not possible for one to be their own brother.
    \item \(R\) is not symmetric because if \(a\in S\) is female and \(b\in S\) is male, then \((a,b)\in R\) but \((b,a)\not\in R\) because \(a\) is 
    \(b\)'s sister.
    \item \(R\) is not transitive because if \(a,b\in S\) are brothers, then \((a,b)\in R\) and \((b,a)\in R\). This implies according to transitivity that
    \((a,a)\in R\), but this is false because you cannot be your own brother. Therefore, \(R\) is not transitive. 
\end{itemize}

Let \(S=\) the set of people. Let \(R=\) ``is sibling of.'' This relation is not reflexive or transitive, but it is symmetric:
\begin{itemize}
    \item \(R\) is not reflexive because it is impossible for one to be their own sibling.
    \item \(R\) is symmetric because if \(a,b\in S\) are siblings, then \((a,b)\in R\) and \((b,a)\in R\).
    \item \(R\) is not transitive because if \(a,b\in S\), then \((a,b)\in R\) and \((b,a)\in R\). Transitivity would imply \((a,a)\in R\), but again \(a\)
    cannot be their own sibling. Therefore, \(R\) is not transitive. 
\end{itemize}

\subsubsection{Graphical Examples}
\begin{figure}[H]
    \centering
    \textbf{Example of non-reflexive, non-symmetric, non-transitive relation}

    \resizebox{12em}{12em}{
        \begin{tikzpicture}[line cap=round,line join=round,>=triangle 45,x=1.0cm,y=1.0cm]
            \clip(-2.,-7.) rectangle (12.,7.);
            \draw [line width=2.pt] (0.,0.) circle (1.cm);
            \draw [line width=2.pt] (5.,5.) circle (1.cm);
            \draw [line width=2.pt] (5.,-5.) circle (1.cm);
            \draw [line width=2.pt] (10.,0.) circle (1.cm);
            \draw [->,line width=2.pt, shorten >=3.pt, shorten <=3.pt] (5.,4.)-- (9.,0.);
            \draw [->,line width=2.pt, shorten >=3.pt, shorten <=3.pt] (9.,0.)-- (5.,-4.);
            \draw (4.6,5.4) node[anchor=north west] {\LARGE$A$};
            \draw (9.6,0.4) node[anchor=north west] {\LARGE$B$};
            \draw (4.6,-4.7) node[anchor=north west] {\LARGE$C$};
            \draw (-0.3,0.4) node[anchor=north west] {\LARGE$D$};
        \end{tikzpicture}
    }
\end{figure}
\begin{figure}[H]
    \centering 
    \textbf{Example of non-reflexive, non-symmetric, transitive relation}

    \resizebox{12em}{12em}{
        \begin{tikzpicture}[line cap=round,line join=round,>=triangle 45,x=1.0cm,y=1.0cm]
            \clip(-2.,-7.) rectangle (12.,7.);
            \draw [line width=2.pt] (0.,0.) circle (1.cm);
            \draw [line width=2.pt] (5.,5.) circle (1.cm);
            \draw [line width=2.pt] (5.,-5.) circle (1.cm);
            \draw [line width=2.pt] (10.,0.) circle (1.cm);
            \draw [->,line width=2.pt, shorten >=3.pt, shorten <=3.pt] (5.,4.)-- (9.,0.);
            \draw [->,line width=2.pt, shorten >=3.pt, shorten <=3.pt] (5.,4.)-- (5.,-4.);
            \draw [->,line width=2.pt, shorten >=3.pt, shorten <=3.pt] (9.,0.)-- (5.,-4.);
            \draw (4.6,5.4) node[anchor=north west] {\LARGE$A$};
            \draw (9.6,0.4) node[anchor=north west] {\LARGE$B$};
            \draw (4.6,-4.7) node[anchor=north west] {\LARGE$C$};
            \draw (-0.3,0.4) node[anchor=north west] {\LARGE$D$};
        \end{tikzpicture}
    }
\end{figure}
\begin{figure}[H]
    \centering
    \textbf{Example of non-reflexive, symmetric, non-transitive relation}

    \resizebox{12em}{12em}{
        \begin{tikzpicture}[line cap=round,line join=round,>=triangle 45,x=1.0cm,y=1.0cm]
            \clip(-2.,-7.) rectangle (12.,7.);
            \draw [line width=2.pt] (0.,0.) circle (1.cm);
            \draw [line width=2.pt] (5.,5.) circle (1.cm);
            \draw [line width=2.pt] (5.,-5.) circle (1.cm);
            \draw [line width=2.pt] (10.,0.) circle (1.cm);
            \draw [<->,line width=2.pt, shorten >=3.pt, shorten <=3.pt] (5.,4.)-- (9.,0.);
            \draw [<->,line width=2.pt, shorten >=3.pt, shorten <=3.pt] (9.,0.)-- (5.,-4.);
            \draw (4.6,5.4) node[anchor=north west] {\LARGE$A$};
            \draw (9.6,0.4) node[anchor=north west] {\LARGE$B$};
            \draw (4.6,-4.7) node[anchor=north west] {\LARGE$C$};
            \draw (-0.3,0.4) node[anchor=north west] {\LARGE$D$};
        \end{tikzpicture}
    }
\end{figure}
\begin{figure}[H]
    \centering
    \textbf{Examples of non-reflexive, symmetric, transitive relation}

    \resizebox{12em}{12em}{
        \begin{tikzpicture}[line cap=round,line join=round,>=triangle 45,x=1.0cm,y=1.0cm]
            \clip(-2.,-7.) rectangle (12.,7.);
            \draw [line width=2.pt] (0.,0.) circle (1.cm);
            \draw [line width=2.pt] (5.,5.) circle (1.cm);
            \draw [line width=2.pt] (5.,-5.) circle (1.cm);
            \draw [line width=2.pt] (10.,0.) circle (1.cm);
            \draw (4.6,5.4) node[anchor=north west] {\LARGE$A$};
            \draw (9.6,0.4) node[anchor=north west] {\LARGE$B$};
            \draw (4.6,-4.7) node[anchor=north west] {\LARGE$C$};
            \draw (-0.3,0.4) node[anchor=north west] {\LARGE$D$};
        \end{tikzpicture}
    }
    \hspace*{5em}
    \resizebox{12em}{12em}{
        \begin{tikzpicture}[line cap=round,line join=round,>=triangle 45,x=1.0cm,y=1.0cm]
            \clip(-2.,-7.) rectangle (12.,7.);
            \draw [line width=2.pt] (0.,0.) circle (1.cm);
            \draw [line width=2.pt] (5.,5.) circle (1.cm);
            \draw [line width=2.pt] (5.,-5.) circle (1.cm);
            \draw [line width=2.pt] (10.,0.) circle (1.cm);
            \draw [<->,shorten >=3.pt,shorten <=3.pt,shift={(5.,6.)},line width=2.pt]  plot[domain=-0.5235987755983:3.6651914291880927,variable=\t]({1.*1.*cos(\t r)+0.*1.*sin(\t r)},{0.*1.*cos(\t r)+1.*1.*sin(\t r)});
            \draw [<->,shorten >=3.pt,shorten <=3.pt,shift={(11.,0.)},line width=2.pt]  plot[domain=-2.094395102393194:2.094395102393194,variable=\t]({1.*1.*cos(\t r)+0.*1.*sin(\t r)},{0.*1.*cos(\t r)+1.*1.*sin(\t r)});
            \draw [<->,shorten >=3.pt,shorten <=3.pt,shift={(5.,-6.)},line width=2.pt]  plot[domain=-3.6651914291880927:0.5235987755982998,variable=\t]({1.*1.*cos(\t r)+0.*1.*sin(\t r)},{0.*1.*cos(\t r)+1.*1.*sin(\t r)});
            \draw [<->,shorten >=3.pt,shorten <=3.pt,line width=2.pt] (5.,4.)-- (9.,0.);
            \draw [<->,shorten >=3.pt,shorten <=3.pt,line width=2.pt] (5.,4.)-- (5.,-4.);
            \draw [<->,shorten >=3.pt,shorten <=3.pt,line width=2.pt] (9.,0.)-- (5.,-4.);
            \draw (4.6,5.4) node[anchor=north west] {\LARGE$A$};
            \draw (9.6,0.4) node[anchor=north west] {\LARGE$B$};
            \draw (4.6,-4.7) node[anchor=north west] {\LARGE$C$};
            \draw (-0.3,0.4) node[anchor=north west] {\LARGE$D$};
        \end{tikzpicture}
    }
\end{figure}
\begin{figure}[H]
    \centering
    \textbf{Example of reflexive, symmetric, non-transitive relation}

    \resizebox{12em}{8em}{
        \begin{tikzpicture}[line cap=round,line join=round,>=triangle 45,x=1.0cm,y=1.0cm]
            \clip(-3.,-2.) rectangle (13.,8.);
            \draw [line width=2.pt] (0.,0.) circle (1.cm);
            \draw [line width=2.pt] (5.,5.) circle (1.cm);
            \draw [line width=2.pt] (10.,0.) circle (1.cm);
            \draw [<->,shorten >=3.pt,shorten <=3.pt,shift={(5.,6.)},line width=2.pt]  plot[domain=-0.5235987755983:3.6651914291880927,variable=\t]({1.*1.*cos(\t r)+0.*1.*sin(\t r)},{0.*1.*cos(\t r)+1.*1.*sin(\t r)});
            \draw [<->,shorten >=3.pt,shorten <=3.pt,shift={(11.,0.)},line width=2.pt]  plot[domain=-2.094395102393194:2.094395102393194,variable=\t]({1.*1.*cos(\t r)+0.*1.*sin(\t r)},{0.*1.*cos(\t r)+1.*1.*sin(\t r)});
            \draw [<->,shorten >=3.pt,shorten <=3.pt,shift={(-1.,0.)},line width=2.pt]  plot[domain=1.0471975511965976:5.235987755982989,variable=\t]({1.*1.*cos(\t r)+0.*1.*sin(\t r)},{0.*1.*cos(\t r)+1.*1.*sin(\t r)});
            \draw [<->,shorten >=3.pt,shorten <=3.pt,line width=2.pt] (5.,4.)-- (9.,0.);
            \draw [<->,shorten >=3.pt,shorten <=3.pt,line width=2.pt] (5.,4.)-- (1.,0.);
            \draw (4.6,5.4) node[anchor=north west] {\LARGE$A$};
            \draw (9.6,0.4) node[anchor=north west] {\LARGE$B$};
            \draw (-0.3,0.4) node[anchor=north west] {\LARGE$D$};
        \end{tikzpicture}
    }
\end{figure}

\subsubsection{Back to Definitions}

\begin{definition}
    A relation \(R\) on set \(S\) is an \emph{equivalence relation}\index{Set Theory!Equivalence Relation} if it is reflexive, symmetric, and transitive.
\end{definition}

\bigskip
\begin{definition}
    Given equivalence relation \(R\subseteq S\times S\) of set \(S\), the \emph{equivalence class} of \(a\in A\) is \(\setbuild{x\in A}{(x,a)\in R}\). The elements
    of this set is said to be \emph{equivalent} to \(a\).\footnote{This definition comes from David S. Dummit's \& Richard M. Foote's \emph{Abstract Algebra}}
\end{definition}

\bigskip
\begin{definition}
    A \emph{partition} of \(A\) is any collection \(\setbuild{A_i}{i\in I}\) of nonempty subsets of \(A\) (\(I\) is some indexing set) such that
    \(A=\bigcup_{i\in I}A_i\) and \(A_i\cap A_j=\emptyset\sforall i,j\in I\:i\ne j\). In other words, \(A\) is the disjoint union of all the sets in the partition.
    \footnote{ditto.}
\end{definition}

\bigskip
\begin{proposition}
    Given equivalence relation \(R\), then the set of equivalence classes of \(R\) forms a partition of \(A\).
\end{proposition}
\begin{proof}
    Let \(K=\setbuild{A_i}{i\in I}\) be the set of equivalence classes of equivalence relation \(R\) where \(A_i\) is an equivalence class. Therefore, it needs to be 
    proven that \(A=\bigcup_{i\in I}A_i\) and \(A_i\cap A_j=\emptyset\sforall i,j\in I\:i\ne j\).

    Assume \(A\ne\bigcup_{i\in I}A_i\). Therefore, there exists \(n\in A\) such that \(\forall\:i\in I\:n\not\in A_i\). However, \(R\) is an equivalence relation, meaning
    it is reflexive which implies \((n,n)\in R\). Therefore, \(n\) belongs to the equivalence class \(\setbuild{x\in A}{(x,n)\in R}\). However, \(K\) is defined to contain
    all equivalence classes, meaning \(\exists\:i\in I\: n\in A_i\). This contradicts that \(n\) is not in any equivalence class. Therefore, by contradiction,
    \(A=\bigcup_{i\in I}A_i\).
    
    Assume for some \(i,j\in I\:i\ne j\), \(A_i\cap A_j=P\) where \(P\) is a nonempty set. Therefore, \(\exists\:p\in P\) such that \(p\in A_i\) and \(p\in A_j\). By
    definition of the equivalence classes, \((p,y)\in R\) for any \(y\in A_i\) and \((p,z)\in R\) for any \(z\in A_j\). Applying the symmetric property, \((y,p)\in R\)
    as well. Now applying the transitive property yields \((y,z)\in R\). This means that \(y,z\) are in the same equivalence class. Therefore, \(A_i=A_j\) since all elements
    since all elements between the two are equivalent to each other. However, this contradicts how \(A_i\ne A_j\) since \(i\ne j\). Thus, \(A_i\cap A_j=\emptyset\).

    In conclusion, \(K\) is a partition of \(A\), and it is said that the equivalence relation on \(A\) partitions the set \(A\).
\end{proof}

\bigskip
\begin{proposition}
    If \(\setbuild{A_i}{i\in I}\) is a partition of set \(A\). Then, there exists an equivalence relation \(R\) on set \(S\) whose equivalence classes are \(A_i\sforall i\in I\).
    \footnote{I am not doing this proof.}
\end{proposition}

\subsubsection{Examples}
Proposition \thechapter.2 states that if there exists a partition on a set, then there must exist an equivalence relation. Therefore, equivalence 
relations are common:
\begin{itemize}
    \item Let \(S=\) set of species and binary relation \(R=\)``is the same species as.'' This is an equivalence relation.
    \item Let \(S=\mathbb{Z}\) and the equivalence relation \(R=\)``has the same remainder modulo \(m\).''
    \footnote{This is an ubiquitous example of equivalence relation. More on this later.}
    \item Let \(S=\mathbb{Z}\) and the equivalence relation \(R=\)``equals.''
    \item Let \(S=\) set of all people and equivalence relation \(R=\)``same age as.''
\end{itemize} 

Equivalence relationship formed by remainder when dividing by \(m\) is crucial in many mathematical studies such as number theory.
Modular arithmetic\index{Modular Arithmetic!Modular Congruences} is primarily focused on the binary equivalence relation called modular
congruences.

\begin{definition}
    Given \(a,b\in\mathbb{Z}\), \(a\) is congruent to \(b\) modulo \(m\), i.e.\ \(\modcong{a}{b}{m}\), if \(m\mid a-b\).
\end{definition}

The \(\mid\) symbol means ``divides.'' Therefore, \(\modcong{a}{b}{m}\) if \(m\) divides the difference between \(a\) and \(b\).

\begin{proposition}
    \(m\mid a-b\) is an equivalence relation on \(\mathbb{Z}\).
\end{proposition}
\begin{proof}
    To prove that \(m\mid a-b\) is an equivalence relation \(R\), all the properties need to be proven:
    \begin{itemize}
        \item \(\forall\:a\in\mathbb{Z}\:m\mid a-a\) which is logically equivalent to \(m\mid 0\) which is always true.
        Therefore, \(\forall\:a\in\mathbb{Z}\:(a,a)\in R\). This proves reflexive property.
        \item Given \((a,b)\in R\), then \(m\mid a-b\). Multiplying \(a-b\) by \(-1\) yields \(b-a\). Therefore, \(m\mid b-a\)
        and \((b,a)\in R\sforall a,b\in\mathbb{Z}\). This proves the symmetric property.
        \item Given \((a,b)\in R\) and \((b,c)\in R\). By definition, \(m\mid a-b\) and \(m\mid b-c\). Therefore, \(m\mid(a-b)+(b-c)\)
        which yields \(m\mid a-c\) meaning \((a,c)\in R\sforall a,b,c\in\mathbb{Z}\). This proves the transitive property. 
    \end{itemize}
    Therefore, \(m\mid a-b\) is an equivalence relation.
\end{proof}

\textbf{Note:} The definition of the relation as \(m\mid a-b\) is different from the above definition as having the same remainder.
Thinking more about it, it should be clear that if two numbers \(a,b\) have the same remainder, then \(m\mid a-b\) and it satisfies the 
relation. However, this definition is more mathematical and allows us to perform operations like the ones shown above to prove properties.
Additionally, the idea of modular congruence extends beyond the integers (like into integer polynomials and Gaussian integers). In some
cases, like the Gaussian integers, there are some ambiguity over the definition of remainder, i.e.\ there may be more than one valid 
quotient and remainder. This definition avoids this issue.\footnote{Number theory is very interesting.}

\subsubsection{Graphical Examples}

\begin{figure}[H]
    \centering
    \textbf{Examples of equivalence relation}

    \resizebox{10em}{10em}{
        \begin{tikzpicture}[line cap=round,line join=round,>=triangle 45,x=1.0cm,y=1.0cm]
            \clip(-2.,-7.) rectangle (12.,7.);
            \draw [line width=2.pt] (0.,0.) circle (1.cm);
            \draw [line width=2.pt] (5.,5.) circle (1.cm);
            \draw [line width=2.pt] (5.,-5.) circle (1.cm);
            \draw [line width=2.pt] (10.,0.) circle (1.cm);
            \draw [<->,shorten <=3.pt,shorten >=3.pt,shift={(5.,6.)},line width=2.pt]  plot[domain=-0.5235987755983:3.6651914291880927,variable=\t]({1.*1.*cos(\t r)+0.*1.*sin(\t r)},{0.*1.*cos(\t r)+1.*1.*sin(\t r)});
            \draw [<->,shorten <=3.pt,shorten >=3.pt,shift={(11.,0.)},line width=2.pt]  plot[domain=-2.094395102393194:2.094395102393194,variable=\t]({1.*1.*cos(\t r)+0.*1.*sin(\t r)},{0.*1.*cos(\t r)+1.*1.*sin(\t r)});
            \draw [<->,shorten <=3.pt,shorten >=3.pt,shift={(5.,-6.)},line width=2.pt]  plot[domain=-3.6651914291880927:0.5235987755982998,variable=\t]({1.*1.*cos(\t r)+0.*1.*sin(\t r)},{0.*1.*cos(\t r)+1.*1.*sin(\t r)});
            \draw [<->,shorten <=3.pt,shorten >=3.pt,shift={(-1.,0.)},line width=2.pt]  plot[domain=1.0471975511965976:5.235987755982989,variable=\t]({1.*1.*cos(\t r)+0.*1.*sin(\t r)},{0.*1.*cos(\t r)+1.*1.*sin(\t r)});
            \draw (4.6,5.4) node[anchor=north west] {\LARGE$A$};
            \draw (9.6,0.4) node[anchor=north west] {\LARGE$B$};
            \draw (4.6,-4.7) node[anchor=north west] {\LARGE$C$};
            \draw (-0.3,0.4) node[anchor=north west] {\LARGE$D$};
        \end{tikzpicture}
    }
    \hspace*{3em}
    \resizebox{10em}{10em}{
        \begin{tikzpicture}[line cap=round,line join=round,>=triangle 45,x=1.0cm,y=1.0cm]
            \clip(-2.,-7.) rectangle (12.,7.);
            \draw [line width=2.pt] (0.,0.) circle (1.cm);
            \draw [line width=2.pt] (5.,5.) circle (1.cm);
            \draw [line width=2.pt] (5.,-5.) circle (1.cm);
            \draw [line width=2.pt] (10.,0.) circle (1.cm);
            \draw [<->,shorten >=3.pt,shorten <=3.pt, shift={(5.,6.)},line width=2.pt]  plot[domain=-0.5235987755983:3.6651914291880927,variable=\t]({1.*1.*cos(\t r)+0.*1.*sin(\t r)},{0.*1.*cos(\t r)+1.*1.*sin(\t r)});
            \draw [<->,shorten >=3.pt,shorten <=3.pt, shift={(11.,0.)},line width=2.pt]  plot[domain=-2.094395102393194:2.094395102393194,variable=\t]({1.*1.*cos(\t r)+0.*1.*sin(\t r)},{0.*1.*cos(\t r)+1.*1.*sin(\t r)});
            \draw [<->,shorten >=3.pt,shorten <=3.pt, shift={(5.,-6.)},line width=2.pt]  plot[domain=-3.6651914291880927:0.5235987755982998,variable=\t]({1.*1.*cos(\t r)+0.*1.*sin(\t r)},{0.*1.*cos(\t r)+1.*1.*sin(\t r)});
            \draw [<->,shorten >=3.pt,shorten <=3.pt, shift={(-1.,0.)},line width=2.pt]  plot[domain=1.0471975511965976:5.235987755982989,variable=\t]({1.*1.*cos(\t r)+0.*1.*sin(\t r)},{0.*1.*cos(\t r)+1.*1.*sin(\t r)});
            \draw [<->,shorten >=3.pt,shorten <=3.pt, line width=2.pt] (5.,4.)-- (9.,0.);
            \draw [<->,shorten >=3.pt,shorten <=3.pt, line width=2.pt] (5.,4.)-- (5.,-4.);
            \draw [<->,shorten >=3.pt,shorten <=3.pt, line width=2.pt] (5.,4.)-- (1.,0.);
            \draw [<->,shorten >=3.pt,shorten <=3.pt, line width=2.pt] (1.,0.)-- (9.,0.);
            \draw [<->,shorten >=3.pt,shorten <=3.pt, line width=2.pt] (1.,0.)-- (5.,-4.);
            \draw [<->,shorten >=3.pt,shorten <=3.pt, line width=2.pt] (9.,0.)-- (5.,-4.);
            \draw (4.6,5.4) node[anchor=north west] {\LARGE$A$};
            \draw (9.6,0.4) node[anchor=north west] {\LARGE$B$};
            \draw (4.6,-4.7) node[anchor=north west] {\LARGE$C$};
            \draw (-0.3,0.4) node[anchor=north west] {\LARGE$D$};
        \end{tikzpicture}
    }

    \resizebox{10em}{10em}{
        \begin{tikzpicture}[line cap=round,line join=round,>=triangle 45,x=1.0cm,y=1.0cm]
            \clip(-2.,-7.) rectangle (12.,7.);
            \draw [line width=2.pt] (0.,0.) circle (1.cm);
            \draw [line width=2.pt] (5.,5.) circle (1.cm);
            \draw [line width=2.pt] (5.,-5.) circle (1.cm);
            \draw [line width=2.pt] (10.,0.) circle (1.cm);
            \draw [<->,shorten <=3.pt,shorten >=3.pt,shift={(5.,6.)},line width=2.pt]  plot[domain=-0.5235987755983:3.6651914291880927,variable=\t]({1.*1.*cos(\t r)+0.*1.*sin(\t r)},{0.*1.*cos(\t r)+1.*1.*sin(\t r)});
            \draw [<->,shorten <=3.pt,shorten >=3.pt,shift={(11.,0.)},line width=2.pt]  plot[domain=-2.094395102393194:2.094395102393194,variable=\t]({1.*1.*cos(\t r)+0.*1.*sin(\t r)},{0.*1.*cos(\t r)+1.*1.*sin(\t r)});
            \draw [<->,shorten <=3.pt,shorten >=3.pt,shift={(5.,-6.)},line width=2.pt]  plot[domain=-3.6651914291880927:0.5235987755982998,variable=\t]({1.*1.*cos(\t r)+0.*1.*sin(\t r)},{0.*1.*cos(\t r)+1.*1.*sin(\t r)});
            \draw [<->,shorten <=3.pt,shorten >=3.pt,shift={(-1.,0.)},line width=2.pt]  plot[domain=1.0471975511965976:5.235987755982989,variable=\t]({1.*1.*cos(\t r)+0.*1.*sin(\t r)},{0.*1.*cos(\t r)+1.*1.*sin(\t r)});
            \draw [<->,shorten <=3.pt,shorten >=3.pt,line width=2.pt] (5.,4.)-- (9.,0.);
            \draw [<->,shorten <=3.pt,shorten >=3.pt,line width=2.pt] (5.,4.)-- (5.,-4.);
            \draw [<->,shorten <=3.pt,shorten >=3.pt,line width=2.pt] (9.,0.)-- (5.,-4.);
            \draw (4.6,5.4) node[anchor=north west] {\LARGE$A$};
            \draw (9.6,0.4) node[anchor=north west] {\LARGE$B$};
            \draw (4.6,-4.7) node[anchor=north west] {\LARGE$C$};
            \draw (-0.3,0.4) node[anchor=north west] {\LARGE$D$};
        \end{tikzpicture}
    }
\end{figure}

\subsection{More Properties of Relations}

\begin{definition}
    A relation \(R\subseteq S\times S\) on set \(S\) is \emph{asymmetric} if \(\forall\:x,y\in S\:(x,y)\in R\Rightarrow(y,x)\not\in R\).
\end{definition}

Intuitively, the relation is one-directional. Additionally, a relation that is asymmetric cannot be reflexive (having self-loops) as asymmetry
implies if \((a,a)\in S\) then \((a,a)\not\in S\).

\bigskip
\begin{definition}
    A relation \(R\subseteq S\times S\) on set \(S\) is \emph{antisymmetric} if \(\forall\:x,y\in S\:(x\ne y)\land(x,y)\in R\Rightarrow(y,x)\not\in R\).
\end{definition}

Intuitively, the relation is one-directional, but also allows for self-loops. An antisymmetric relation can be reflexive since
if \((a,b)\in R\) and \((b,a)\in R\), antisymmetry would imply that \(a=b\). Additionally, asymmetry implies antisymmetry.

\subsubsection{Graphical Examples}

\begin{figure}[H]
    \centering
    \textbf{Example of Transitive and Asymmetric Relation}

    \resizebox{12em}{12em}{
        \begin{tikzpicture}[line cap=round,line join=round,>=triangle 45,x=1.0cm,y=1.0cm]
            \clip(-2.,-7.) rectangle (12.,7.);
            \draw [->,shorten <=3.pt,shorten >=3.pt,line width=2.pt] (0.,0.) circle (1.cm);
            \draw [->,shorten <=3.pt,shorten >=3.pt,line width=2.pt] (5.,5.) circle (1.cm);
            \draw [->,shorten <=3.pt,shorten >=3.pt,line width=2.pt] (5.,-5.) circle (1.cm);
            \draw [->,shorten <=3.pt,shorten >=3.pt,line width=2.pt] (10.,0.) circle (1.cm);
            \draw [->,shorten <=3.pt,shorten >=3.pt,line width=2.pt] (5.,4.)-- (9.,0.);
            \draw [->,shorten <=3.pt,shorten >=3.pt,line width=2.pt] (5.,4.)-- (5.,-4.);
            \draw [->,shorten <=3.pt,shorten >=3.pt,line width=2.pt] (5.,4.)-- (1.,0.);
            \draw [->,shorten <=3.pt,shorten >=3.pt,line width=2.pt] (1.,0.)-- (5.,-4.);
            \draw [->,shorten <=3.pt,shorten >=3.pt,line width=2.pt] (9.,0.)-- (5.,-4.);
            \draw (4.6,5.4) node[anchor=north west] {\LARGE$A$};
            \draw (9.6,0.4) node[anchor=north west] {\LARGE$B$};
            \draw (4.6,-4.7) node[anchor=north west] {\LARGE$C$};
            \draw (-0.3,0.4) node[anchor=north west] {\LARGE$D$};
        \end{tikzpicture}
    }
\end{figure}
\begin{figure}[H]
    \centering
    \textbf{Example of Transitive and Antisymmetric Relation}

    \resizebox{12em}{12em}{
        \begin{tikzpicture}[line cap=round,line join=round,>=triangle 45,x=1.0cm,y=1.0cm]
            \clip(-3.,-8.) rectangle (13.,7.);
            \draw [->,shorten <=3.pt,shorten >=3.pt,line width=2.pt] (0.,0.) circle (1.cm);
            \draw [->,shorten <=3.pt,shorten >=3.pt,line width=2.pt] (5.,5.) circle (1.cm);
            \draw [->,shorten <=3.pt,shorten >=3.pt,line width=2.pt] (5.,-5.) circle (1.cm);
            \draw [->,shorten <=3.pt,shorten >=3.pt,line width=2.pt] (10.,0.) circle (1.cm);
            \draw [<->,shorten <=3.pt,shorten >=3.pt,shift={(11.,0.)},line width=2.pt]  plot[domain=-2.094395102393194:2.094395102393194,variable=\t]({1.*1.*cos(\t r)+0.*1.*sin(\t r)},{0.*1.*cos(\t r)+1.*1.*sin(\t r)});
            \draw [<->,shorten <=3.pt,shorten >=3.pt,shift={(5.,-6.)},line width=2.pt]  plot[domain=-3.6651914291880927:0.5235987755982998,variable=\t]({1.*1.*cos(\t r)+0.*1.*sin(\t r)},{0.*1.*cos(\t r)+1.*1.*sin(\t r)});
            \draw [<->,shorten <=3.pt,shorten >=3.pt,shift={(-1.,0.)},line width=2.pt]  plot[domain=1.0471975511965976:5.235987755982989,variable=\t]({1.*1.*cos(\t r)+0.*1.*sin(\t r)},{0.*1.*cos(\t r)+1.*1.*sin(\t r)});
            \draw [->,shorten <=3.pt,shorten >=3.pt,line width=2.pt] (5.,4.)-- (9.,0.);
            \draw [->,shorten <=3.pt,shorten >=3.pt,line width=2.pt] (5.,4.)-- (5.,-4.);
            \draw [->,shorten <=3.pt,shorten >=3.pt,line width=2.pt] (5.,4.)-- (1.,0.);
            \draw [->,shorten <=3.pt,shorten >=3.pt,line width=2.pt] (1.,0.)-- (5.,-4.);
            \draw [->,shorten <=3.pt,shorten >=3.pt,line width=2.pt] (9.,0.)-- (5.,-4.);
            \draw (4.6,5.4) node[anchor=north west] {\LARGE$A$};
            \draw (9.6,0.4) node[anchor=north west] {\LARGE$B$};
            \draw (4.6,-4.7) node[anchor=north west] {\LARGE$C$};
            \draw (-0.3,0.4) node[anchor=north west] {\LARGE$D$};
        \end{tikzpicture}
    }
\end{figure}

\subsection{Partial and Total Order}
\index{Set Theory!Partial Order}
\index{Set Theory!Total Order}

\begin{definition}
    A \emph{partial order} is a relation \(R\) on set \(S\) that is transitive, reflexive, and antisymmetric. 
\end{definition}

Partial orders may also called \emph{weak partial order}, \emph{reflexive partial order}, or \emph{non-strict partial order}.
A simple example of partial order is the relation \(R=\)``greater than or equal to'' on set \(S=\mathbb{Z}\). Reflexivity is
satisfied because \(\forall\:x\in\mathbb{Z}\:x\ge x\). Antisymmetry is satisfied as 
\(\forall\:x,y\in\mathbb{Z}\:x\ne y\land(x\ge y)\Rightarrow\lnot(y\ge x)\). Additionally, transitivity is satisfied as
\(\forall\:x,y,z\in\mathbb{Z}\:(x\ge y)\land(y\ge z)\Rightarrow(x\ge z)\). This concludes that relation \(R\) is partially
ordered.

\bigskip
\begin{definition}
    A \emph{strict partial order} is a relation \(R\) on set \(S\) that is transitive and asymmetric.    
\end{definition}

A simple example of strict partial order is with relation \(R=\)``less than'' on set \(S=\mathbb{Z}\). The proof is similar to the
above, but modified to fit the properties of transitivity and asymmetry. 

\bigskip
\begin{definition}
    A \emph{strict total order} \(\prec\) is a strict partial order where \(\forall\:x,y\in S\:(x\ne y)\Rightarrow(x\prec y)\lor(y\prec x)\).
\end{definition}

Intuitively, in the graph diagram, all nodes must be connected to the other nodes in either direction. Additionally, the nodes can 
be rearranged into a line where the order is defined as the position of a node relative to another.

\subsubsection{Graphical Examples}

\begin{figure}[H]
    \centering
    \textbf{Example of Strict Partial Order}

    \resizebox{12em}{12em}{
        \begin{tikzpicture}[line cap=round,line join=round,>=triangle 45,x=1.0cm,y=1.0cm]
            \clip(-2.,-7.) rectangle (12.,7.);
            \draw [->,shorten <=3.pt,shorten >=3.pt,line width=2.pt] (0.,0.) circle (1.cm);
            \draw [->,shorten <=3.pt,shorten >=3.pt,line width=2.pt] (5.,5.) circle (1.cm);
            \draw [->,shorten <=3.pt,shorten >=3.pt,line width=2.pt] (5.,-5.) circle (1.cm);
            \draw [->,shorten <=3.pt,shorten >=3.pt,line width=2.pt] (10.,0.) circle (1.cm);
            \draw [->,shorten <=3.pt,shorten >=3.pt,line width=2.pt] (5.,4.)-- (9.,0.);
            \draw [->,shorten <=3.pt,shorten >=3.pt,line width=2.pt] (5.,4.)-- (5.,-4.);
            \draw [->,shorten <=3.pt,shorten >=3.pt,line width=2.pt] (5.,4.)-- (1.,0.);
            \draw [->,shorten <=3.pt,shorten >=3.pt,line width=2.pt] (1.,0.)-- (5.,-4.);
            \draw [->,shorten <=3.pt,shorten >=3.pt,line width=2.pt] (9.,0.)-- (5.,-4.);
            \draw (4.6,5.4) node[anchor=north west] {\LARGE$A$};
            \draw (9.6,0.4) node[anchor=north west] {\LARGE$B$};
            \draw (4.6,-4.7) node[anchor=north west] {\LARGE$C$};
            \draw (-0.3,0.4) node[anchor=north west] {\LARGE$D$};
        \end{tikzpicture}
    }
\end{figure}
\begin{figure}[H]
    \centering
    \textbf{Example of Weak Partial Order}

    \resizebox{12em}{12em}{
        \begin{tikzpicture}[line cap=round,line join=round,>=triangle 45,x=1.0cm,y=1.0cm]
            \clip(-3.,-8.) rectangle (13.,8.);
            \draw [->,shorten <=3.pt,shorten >=3.pt,line width=2.pt] (0.,0.) circle (1.cm);
            \draw [->,shorten <=3.pt,shorten >=3.pt,line width=2.pt] (5.,5.) circle (1.cm);
            \draw [->,shorten <=3.pt,shorten >=3.pt,line width=2.pt] (5.,-5.) circle (1.cm);
            \draw [->,shorten <=3.pt,shorten >=3.pt,line width=2.pt] (10.,0.) circle (1.cm);
            \draw [<->,shorten <=3.pt,shorten >=3.pt,shift={(5.,6.)},line width=2.pt]  plot[domain=-0.5235987755983:3.6651914291880927,variable=\t]({1.*1.*cos(\t r)+0.*1.*sin(\t r)},{0.*1.*cos(\t r)+1.*1.*sin(\t r)});
            \draw [<->,shorten <=3.pt,shorten >=3.pt,shift={(11.,0.)},line width=2.pt]  plot[domain=-2.094395102393194:2.094395102393194,variable=\t]({1.*1.*cos(\t r)+0.*1.*sin(\t r)},{0.*1.*cos(\t r)+1.*1.*sin(\t r)});
            \draw [<->,shorten <=3.pt,shorten >=3.pt,shift={(5.,-6.)},line width=2.pt]  plot[domain=-3.6651914291880927:0.5235987755982998,variable=\t]({1.*1.*cos(\t r)+0.*1.*sin(\t r)},{0.*1.*cos(\t r)+1.*1.*sin(\t r)});
            \draw [<->,shorten <=3.pt,shorten >=3.pt,shift={(-1.,0.)},line width=2.pt]  plot[domain=1.0471975511965976:5.235987755982989,variable=\t]({1.*1.*cos(\t r)+0.*1.*sin(\t r)},{0.*1.*cos(\t r)+1.*1.*sin(\t r)});
            \draw [->,shorten <=3.pt,shorten >=3.pt,line width=2.pt] (5.,4.)-- (9.,0.);
            \draw [->,shorten <=3.pt,shorten >=3.pt,line width=2.pt] (5.,4.)-- (5.,-4.);
            \draw [->,shorten <=3.pt,shorten >=3.pt,line width=2.pt] (5.,4.)-- (1.,0.);
            \draw [->,shorten <=3.pt,shorten >=3.pt,line width=2.pt] (1.,0.)-- (5.,-4.);
            \draw [->,shorten <=3.pt,shorten >=3.pt,line width=2.pt] (9.,0.)-- (5.,-4.);
            \draw (4.6,5.4) node[anchor=north west] {\LARGE$A$};
            \draw (9.6,0.4) node[anchor=north west] {\LARGE$B$};
            \draw (4.6,-4.7) node[anchor=north west] {\LARGE$C$};
            \draw (-0.3,0.4) node[anchor=north west] {\LARGE$D$};
        \end{tikzpicture}
    }
\end{figure}
\begin{figure}[H]
    \centering
    \textbf{Example of Strict Total Order}

    \resizebox{12em}{12em}{
        \begin{tikzpicture}[line cap=round,line join=round,>=triangle 45,x=1.0cm,y=1.0cm]
            \clip(-2.,-7.) rectangle (12.,7.);
            \draw [->,shorten <=3.pt,shorten >=3.pt,line width=2.pt] (0.,0.) circle (1.cm);
            \draw [->,shorten <=3.pt,shorten >=3.pt,line width=2.pt] (5.,5.) circle (1.cm);
            \draw [->,shorten <=3.pt,shorten >=3.pt,line width=2.pt] (5.,-5.) circle (1.cm);
            \draw [->,shorten <=3.pt,shorten >=3.pt,line width=2.pt] (10.,0.) circle (1.cm);
            \draw [->,shorten <=3.pt,shorten >=3.pt,line width=2.pt] (5.,4.)-- (9.,0.);
            \draw [->,shorten <=3.pt,shorten >=3.pt,line width=2.pt] (5.,4.)-- (5.,-4.);
            \draw [->,shorten <=3.pt,shorten >=3.pt,line width=2.pt] (5.,4.)-- (1.,0.);
            \draw [->,shorten <=3.pt,shorten >=3.pt,line width=2.pt] (1.,0.)-- (9.,0.);
            \draw [->,shorten <=3.pt,shorten >=3.pt,line width=2.pt] (1.,0.)-- (5.,-4.);
            \draw [->,shorten <=3.pt,shorten >=3.pt,line width=2.pt] (9.,0.)-- (5.,-4.);
            \draw (4.6,5.4) node[anchor=north west] {\LARGE$A$};
            \draw (9.6,0.4) node[anchor=north west] {\LARGE$B$};
            \draw (4.6,-4.7) node[anchor=north west] {\LARGE$C$};
            \draw (-0.3,0.4) node[anchor=north west] {\LARGE$D$};
        \end{tikzpicture}
    }

    In this example, the order of the nodes on a line from left to right would be \(A\prec D\prec B\prec C\).
\end{figure}